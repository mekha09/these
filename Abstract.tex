\chapternonum{Abstract}
When a low-frequency laser pulse is focused to a high intensity into a gas, the electric field of the laser light may become of comparable strength to that felt by the electrons bound in an atom or molecule. A valence electron can then be 'freed' by tunnel ionization, accelerated by the strong oscillating laser field and can eventually recollide and recombine with the ion. The gained kinetic energy is then released as a burst of coherent XUV light and the macroscopic gas medium then becomes a source of XUV light pulses of attosecond (1 as = 10$^{-18}\:$s) duration. This is the natural time-scale of electron dynamics in atoms and molecules.

%In this thesis, we study this process, known as `High Harmonic Generation', with regard to probing, or even imaging electrons and their dynamics in molecules. This can be tackled in essentially two different schemes: (i) The molecules are the harmonic generation medium and the recolliding electron wave packet acts as a self-probe after an attosecond delay given by the time between ionization and recombination. The XUV light emission then carries information about the bound state wave function. (ii) The coherent XUV light emitted by rare gas atoms is a well characterized source that can be used to photoionize molecules. Measuring the ejected photoelectron wave packet then allows to extract information on the photoionization process itself, and possibly about the intial bound and final continuum state of the electron.

The largest part of this thesis deals with experiments where molecules are the harmonic generation medium and the recolliding electron wave packet acts as a `self-probe'. In several experiments, we demonstrate the potential of this scheme to observe or image ultra-fast intra-molecular electronic and nuclear dynamics. In particular, we have performed the first phase measurements of the high harmonic emission from aligned molecules. From measurements characterizing in amplitude and phase the high harmonic emission from CO$_2$ and N$_2$ molecules aligned in the laboratory frame, we extract the recombination dipole matrix element, i.e. the probability \emph{amplitude} for the continuum electron to recombine into the bound state. This observable contains signatures of quantum interference between the continuum and bound parts of the total electronic wavefunction. It is shown how this quantum interference can be utilized to shape the attosecond light emission from the molecules. Furthermore, a set of recombination dipole matrix elements for electron-ion recollision directions from 0$^\circ$ to 360$^\circ$ may contain sufficient information to reconstruct the bound-state \emph{wavefunction} using a tomographic algorithm. The theoretical basis of this method of molecular orbital tomography is examined, the technique's potential and limitations are presented and the experimental feasibility is demonstrated. This opens the perspective of imaging ultra-fast changes of, e.g., a frontier orbital during a chemical reaction.

In a second part of this thesis, we use the well characterized coherent XUV light emitted by rare gas atoms to photoionize molecules. Measuring the ejected photoelectron wave packet then allows to extract information on the photoionization process itself, and possibly about the initial bound and final continuum states of the electron. We measure how an auto-ionizing resonance in N$_2$ molecules modifies the spectral phase of the photoelectron wavepacket.

The last chapter of this manuscript describes studies of high harmonic and attosecond light pulse generation in a different medium: ablation plasmas. We perform the first temporal characterization of such a source, demonstrating the femtosecond and attosecond structure of the emitted XUV intensity profile. 