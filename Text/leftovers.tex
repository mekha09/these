Le moment angulaire est à la rotation ce que la quantité de mouvement est à la translation. C'est une grandeur fondamentale en physique car elle est \textit{conservée}. Cette quantité peut être définie pour un satellite, une galaxie, une molécule, ou encore un champ. Dans cette thèse, nous nous intéressons au cas du champ électromagnétique : plus précisément, on manipulera le moment angulaire du champ infrarouge produit par un laser, ou bien du rayonnement ultraviolet constitué des harmoniques d'ordres élevées décrit dans la partie précédente.\par




