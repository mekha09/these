\chapternonumtoc{Introduction}
%
%Les travaux présentés dans cette thèse s’inscrivent dans le cadre de l’interaction laser-matière. Les échanges d’énergie, d’impulsion ou de moment angulaire, entre le champ électromagnétique et la matière sous toutes ses formes, sont au cœur du monde physique. Le fort couplage lumière-matière permet, d’une part, à travers la spectroscopie, d’étudier les propriétés structurales des systèmes en phase gazeuse, liquide, solide ou plasma.
%Il permet, d’autre part, d’étudier des processus dynamiques – excitation électronique, mouvement nucléaire, réactivité chimique, transition de phase, etc.
%En exploitant la conjugaison temps-énergie, la spectroscopie a longtemps été la seule technique à fournir des informations précieuses sur les dynamiques les plus rapides. Le profil spectral, mesuré en photoémission ou en photo-absorption au voisinage d’un état excité, renseigne assez directement sur le temps caractéristique d’évolution (la durée de vie) de cet état. La spectroscopie devient cependant très difficile à mettre en œuvre, et de plus incomplète, quand l’état excité transitoire du système étudié se projette sur un très grand nombre d’états stationnaires (par exemple, dans une molécule, plusieurs continuums de dissociation ou d’ionisation), les projections (coefficients complexes) ayant de plus des phases spectrales non nulles inaccessibles à la spectroscopie standard.
%Depuis les années 60 et l’avènement du laser, les études « en temps réel » des processus dynamiques ont pris une importance croissante : elles considèrent « globalement » l’état excité transitoire du système, correspondant à un paquet d’ondes, dont on peut suivre la trajectoire partiellement localisée dans l’espace et le temps en « sondant » le système. Les études résolues en temps complètent et approfondissent considérablement le champ ouvert par la spectroscopie. On notera que, très souvent, les études spectroscopiques et « résolues en temps » combinent mutuellement leurs schémas d’excitation et leurs techniques de détection, une autre preuve de leur richesse et de leur complémentarité.
%Finalement, l’interaction laser-matière – notamment en champ laser intense, soit dans des régimes fortement non linéaires – donne lieu à un grand nombre de processus spécifiques et spectaculaires, tels que l’ionisation au-dessus du seuil et la diffraction d’électrons, l’émission harmonique dans les gaz, les solides et les plasmas, l’accélération de particules chargées. Les processus en champ fort constituent en eux-mêmes un vaste domaine d’étude, aujourd’hui en plein essor auprès de grandes à très grandes installations laser (ex. le laser APOLLON).
%L’extension et la diversité des études qui relèvent de l’interaction lumière-matière doivent beaucoup au développement de la technologie laser.  L'avènement du laser (\mycite{SchawlowPR1958, MaimanNature1960}) au début des années 1960 a permis l'essor de la spectroscopie à haute-résolution, qui a révélé systématiquement les structures rotationnelles, vibrationnelles et électroniques de molécules (\mycite{HerzbergMolecule}). Les lasers sont également devenus très rapidement un outil idéal pour étudier "en temps réel" les processus dynamiques ultrarapides. Les énormes progrès technologiques réalisés en une quarantaine d'années ont notamment permis de réduire la durée des impulsions (\mycite{AgostiniRPP2004}). Les premiers lasers impulsionnels produisaient des impulsions de l'ordre de la centaine de microsecondes. Quelques années plus tard, l'invention du "Q-switching" par \mycite{HellwarthAQE1961, McClungAO1962} a permis de diminuer la durée des impulsions de quatre ordres de grandeur, jusqu'à l'échelle de la dizaine de nanosecondes. La technique de "mode-locking" (\mycite{DeMariaAPL1966}) combinée à un milieu laser à gain large-bande (ex. laser à colorant, \mycite{ShankAPL1974}) a encore réduit la durée des impulsions de 4 ordres de grandeur, jusqu'à moins d'une picoseconde. L'échelle femtoseconde a été atteinte par \mycite{ForkOL1987}, grâce au développement d'une cavité laser en anneau avec compensation de la dispersion: des impulsions de $6fs$ ont alors été produites. Finalement, depuis une vingtaine d'années, les lasers basés sur une architecture Titane:Saphir permettent d'atteindre des durées d'impulsions proches du cycle optique, soit $2.7fs$ à $800nm$. 
%
%Grâce à ces avancées technologiques, les études résolues en temps de processus dynamiques "modèles" sont possibles dans différents systèmes, via des expériences de type pompe/sonde. Dans ces expériences, la première impulsion de pompe vient exciter le système à étudier: elle crée un paquet d’ondes électronique ou nucléaire \textit{cohérent} qui évolue ensuite librement à partir de l’instant d’excitation (c’est en tant qu’il est cohérent que le processus étudié dans une expérience pompe-sonde peut être dit « modèle », les dynamiques « naturelles » n’étant en général que partiellement cohérentes). La seconde impulsion vient sonder en temps réel la dynamique du système à différents instants. Les travaux de \mycite{ZewailJPCA2000}, qui réunissent de nombreuses études résolues en temps de réactions chimiques, fondent la femtochimie et ont été récompensés par un prix Nobel en 1999. Les échelles de temps pertinentes pour l'étude résolue en temps de systèmes moléculaires et de leur réactivité chimique s'étendent au-delà de l'échelle femtoseconde. Dans un système moléculaire, les atomes (noyaux) "tournent" sur une échelle de temps de l'ordre de la picoseconde, vibrent sur une échelle de temps de l'ordre de plusieurs femtosecondes. Plus de mille fois plus légers que les protons, les électrons, ou paquet d'ondes électroniques, oscillent et se déplacent sur une échelle de temps attoseconde ($10^{-18}s$, un électron d’énergie 1 eV se déplace sur la distance caractéristique de 1 angström en une centaine d’attosecondes). Etudier « en temps réel » le mouvement des électrons dans des expériences pompe-sonde demande donc de disposer d’impulsions de lumière – plus généralement de gradients temporels – à l’échelle attoseconde, parfaitement synchronisées entre elles.   
%
%\subsubsection{Génération d'harmoniques d'ordre élevé}
%
%Pour produire des impulsions attosecondes, un problème technique apparaît. Dans le cas d'une impulsion de durée $100as$, la relation $\Delta \omega \Delta t=4\ln2$ impose que le spectre ait une largeur supérieure ou égale à $18eV$ (égale si l'impulsion est "limitée par transformée de Fourier"). Il faut alors que l'énergie centrale soit au moins de $9eV$, dans la région du vide ultra-violet (VUV:$10-100nm$)/extrême ultra-violet (XUV:$1-100nm$). Une façon de produire des spectres aussi étendus repose sur la génération d'harmoniques d'ordre élevé (GHOE), en particulier dans les gaz. Ce processus fortement non linéaire est accessible uniquement grâce à l'augmentation significative de l'énergie des impulsions laser (voir la revue de \mycite{BrabecRMP2000}), permettant d'atteindre des éclairements de l'ordre de $10^{15} W.cm^{-2}$. Dans ces conditions, le champ électrique est comparable au champ électrostatique vu par un électron en couche externe. 
%
%La GHOE dans les gaz a été découverte dans les années 80, presque simultanément par \mycite{FerrayJPB1988} au CEA-Saclay et \mycite{McPhersonJOSAB1987} à l'université de Rochester. Ce processus hautement non-linéaire prend place dans l'interaction entre un champ électrique intense, en général polarisé linéairement, et un ensemble d'atomes ou de molécules en phase gazeuse. Le champ laser de génération, à la fréquence fondamentale, induit une polarisation non linéaire du gaz qui rayonne les harmoniques impaires de la fréquence fondamentale, dans le domaine XUV. Le spectre de l'émission peut s'étendre sur plusieurs dizaines ou centaines d'électron-volts, voire jusqu'au $keV$ (\mycite{PopmintchevS2012}). Le spectre est composé de trois parties: une région dite perturbative, où l'intensité des harmoniques produites diminue très rapidement, une région appelée "plateau" où l'intensité des harmoniques est constante et finalement la coupure, où l'intensité décroît rapidement jusqu'à extinction complète du signal. La GHOE dans les gaz peut être décrite par un modèle semi-classique dit "en trois étapes" (\mycite{CorkumPRL1993, SchaferPRL1993, LewensteinPRA1994}): ionisation tunnel de l'atome dans le champ laser, accélération de l'électron éjecté et recombinaison radiative de l'électron avec le coeur ionique, l'énergie en excès étant libérée sous la forme d'une impulsion de lumière attoseconde dans l'XUV.  Dans un champ laser qui compte plusieurs cycles optiques, l'émission harmonique correspond à un train d'impulsions attosecondes espacées d'une demi-période optique, parfaitement synchrone avec le champ fondamental (\mycite{HentschelNature2001, PaulScience2001, MairesseScience2003, TzallasNature2003}). On sait également produire des impulsions attosecondes isolées par différentes techniques (\mycite{SansoneScience2006}). Les caractéristiques de l'émission XUV provenant de la GHOE ont été intensivement étudiées au cours des années 90, afin notamment d'améliorer le rendement de conversion IR/XUV. L'émission est cohérente à la fois spatialement et temporellement, au sens où la phase du champ XUV varie régulièrement dans l'espace et le temps; elle diverge peu. Sa polarisation est, en général, déterminée par celle du champ fondamental, mais aussi par le système générateur comme nous le verrons dans notre travail.
%
%\subsubsection{Physique attoseconde}
%
%Aujourd'hui, la physique attoseconde comporte deux directions majeures (\mycite{SalieresRPP2012}). La première direction, transpose les techniques de la physique femtoseconde à l'échelle de l'attoseconde.  Dans un schéma pompe/sonde qui combine plusieurs types d'impulsions, les impulsions XUV attosecondes -isolées ou sous forme de train- peuvent être utilisées pour pomper ou sonder le système étudié (atomes et molécules en phase gazeuse, liquide ou solide) en l'excitant dans un état lié ou dans un continuum (ionisation, dissociation). Elles peuvent également servir de sonde, en photoémission ou en photoabsorption. Quand la sonde (ou la pompe) est du type photoémission, plusieurs technique -RABBIT (voir ci-dessous), attosecond streaking (\mycite{ItataniPRL2002}), FROG-CRAB (\mycite{MairessePRA2005})- donnent accès à l'amplitude et à la phase spectrale du paquet d'ondes électroniques émis, d'où l'on peut extraire l'information sur la dynamique intra-atomique/moléculaire. Les techniques ci-dessus ont servi aux études de dynamiques attosecondes dans des systèmes variés (\mycite{SansoneNature2010, CalegariScience2014, SolaNP2006, UiberackerNature2007, EckleScience2008, MauritssonPRL2008}). Dans les études résolues en temps comme dans la spectroscopie moléculaire, la technique RABBIT (Reconstruction of Attosecond Beating by Interference of Two-photon Transitions), de caractérisation en amplitude et en phase spectrale d’un paquet d’ondes électronique, joue un rôle central. Cette technique a été initialement développée pour mesurer la phase spectrale de l’émission harmonique, permettant la reconstruction dans le domaine temporel du train d’impulsions attoseconde (Paul et al. [38]). Plus généralement, elle s’étend à la caractérisation des paquets d’ondes électroniques produits dans la photoionisation à deux photons XUV-laser d’un système atomique/moléculaire en phase gazeuse. Par exemple, dans le cas de la photoémission dans un continuum "plat", \textit{i.e} sans résonances, la technique RABBIT permet l'extraction des délais de photoémission du paquet d'ondes électroniques qui diffèrent entre deux canaux d'ionisation (\mycite{KlunderPRL2011}).
%
%La technique RABBIT peut aussi être utilisée pour étudier le "retard" à l'ionisation autour d'une résonance. Une première étude a été effectuée par \mycite{HaesslerPRA2009}, qui ont étudié la photoionisation de la molécule de $N_2$ près d'une résonance d'autoionisation. un déphasage de $0.9\pi rad$ a été observé pour les électrons produits dans les canaux d'ionisation correspondant aux états $X^2\Sigma_g^+$, $v'=1$ et $v'=2$ de l'ion moléculaire. Cette étude a été complétée par des simulations, effectués par \mycite{CaillatPRL2011}. Les auteurs observent que la phase de l'amplitude de transition à deux photons varie significativement autour de la résonance et que la dérivée spectrale de cette phase donne un accès direct au délai de création dans le continuum du paquet d'ondes électronique résonant à 2 photons.
%Cependant, aucune étude expérimentale à ce jour n'a permis de mesurer la phase spectrale d'une résonance. Nous montrerons, au chapitre \ref{CH:FanoHelium} qu'il est possible d'accéder à l'évolution de cette phase en modifiant la longueur d'onde de génération. Nous montrerons également qu'en choisissant avec précision les paramètres expérimentaux, il est possible de mesurer complètement la variation de phase à l'aide d'une unique mesure. 
%Plusieurs études des dynamiques associées aux résonances ont été effectuées à la fois théoriquement (\mycite{TongPRA2005, ZhaoPRA2005, MorishitaPRL2007, WickenhauserPRL2005, ChuPRA2012}) et expérimentalement (\mycite{OttScience2013, OttNature2014, MauritssonPRL2010, GilbertsonPRL2010}). Ces études ont notamment permis de remonter au temps de vie des résonances étudiées, en confirmant les résultats de la spectroscopie. Mais elles n'ont pas permis d'accéder à la dynamique d'ionisation du paquet d'ondes électronique près de la résonance non-perturbée, ce que notre étude permettra de reconstruire.\newline
%
%La seconde direction de la physique attoseconde, développée notamment au CEA-Saclay depuis une dizaine d'années, est appelée "spectroscopie harmonique". 
%Comme la spectroscopie classique, linéaire ou faiblement non linéaire, la spectroscopie harmonique, fortement non linéaire, extrait l'information sur le système qui rayonne en caractérisant complètement, en amplitude, en phase et en état de polarisation, l'émission harmonique induite dans le système par une impulsion laser fondamentale. La spectroscopie harmonique peut jouer le rôle de sonde dans les études résolues en temps évoquées au paragraphe précédent (première direction), la pompe étant fournie indépendamment par une première impulsion. La spectroscopie harmonique peut également être considérée, par elle-même, comme une variante du schéma pompe-sonde, quand on veut étudier notamment les dynamiques induites par l'ionisation en champ fort. Pour mieux comprendre ce rapprochement, reprenons le modèle en trois étapes. La première étape correspond à l'impulsion de pompe qui induit l'ionisation tunnel du système. L'excursion du paquet d'ondes électronique dans le continuum peut être vue comme le délai pompe-sonde, pendant lequel le système ionisé évolue. Finalement, la recombinaison radiative du paquet d'ondes électronique avec l'ion agit comme une sonde ultra-courte. Ce schéma d'"auto-sonde" permet alors d'extraire l'information sur la structure et/ou la dynamique de l'ionisation tunnel et du système ionisé. De nombreuses études ont été effectuées dans ce schéma. Elles révèlent les dynamiques nucléaires rotationnelles (\mycite{JinPRA2012, LevesquePRL2007, VozziAPL2005}), vibrationnelles (\mycite{LiScience2008, WagnerPNAS2006}) ou encore de dissociation (\mycite{WornerNature2010, TehlarChimia2013, HaesslerJPB2009}).
%
%La spectroscopie harmonique permet également d'accéder à la dynamique électronique ultra-rapide, par exemple en suivant l'évolution de la structure spatiale d'une orbitale électronique de valence dans l'ion. La reconstruction d'orbitales moléculaires a été proposée initialement par \mycite{ItataniNature2004}. Elle repose sur le fait que les fonctions d'ondes électroniques des orbitales les plus hautes occupées (HOMO), impliquées dans la GHOE, sont contenues dans le dipôle de recombinaison, avec leur poids relatif. Dans le cas de petites molécules "simples" que l'on sait aligner (diatomiques), on peut alors reconstruire non seulement l'orbitale dépendant du temps dans l'ion moléculaire, mais aussi les orbitales HOMO stationnaires qui ont été ionisées. Plusieurs expériences ont effectivement permis la reconstruction de la HOMO de $N_2$ (\mycite{HaesslerNP2010}) et $CO_2$ (\mycite{VozziNP2011}) à partir de la mesure de l'amplitude et de la phase spectrale de l'émission harmonique provenant de ces molécules.\newline
%
%En relation avec les études résolues en temps (première direction) et la spectroscopie harmonique (seconde direction), des études récentes ont porté sur le développement et la caractérisation de sources XUV harmoniques de polarisations elliptique ou circulaire. Ces développements sont nécessaires, par exemple, pour les expériences de dichroïsme circulaire en photoémission (\mycite{PowisACP2008}), ou encore de dichroïsme circulaire magnétique de rayons-X (\mycite{KfirNP2015}) qui pourront être résolues en temps. Au début de cette thèse, quelques méthodes de production d'impulsions XUV ultra-brèves de polarisation elliptique, à partir de la GHOE, existaient.
%
%Il est par exemple possible de modifier l'état de polarisation après génération, à l'aide d'un ensemble de plusieurs miroirs déphaseurs (\mycite{VodungboOE2011}). Cependant, l'efficacité en sortie est faible (de quelques $\%$) à cause d'une forte absorption. L'autre méthode, plus communément utilisée, est basée sur la brisure de symétrie de la GHOE, soit par le champ fondamental, lui même polarisé elliptiquement \mycite{AntoinePRA1996}, soit par le système générateur (\mycite{ZhouPRL2009, MairessePRL2010}). Dans le cas de la brisure par le champ, il a été montré que l'efficacité de génération diminuait fortement avec l'ellipticité du fondamental (\mycite{BudilPRA1993}). Dans le cas de la GHOE dans l'argon, le signal est complétement éteint pour $\epsilon=30\%$. De plus, le transfert d'ellipticité de l'IR vers l'XUV est linéaire et est a une pente de 1, rendant cette procédure peu attractive.
 %
%Durant cette thèse, un regain d'intérêt pour ce domaine a été observé et de nouvelles méthodes de production d'un rayonnement polarisé elliptiquement, voire quasi-circulairement, ont été développées. Ces méthodes reposent, par exemple, sur l'utilisation d'une résonance de forme dans le processus de GHOE dans un gaz de molécules de $SF_6$ à l'aide d'un champ de génération polarisé elliptiquement (\mycite{ferreNC2015}), permettant l'obtention d'une ellipticité de l'ordre de $80\%$ pour l'harmonique 15 à $800nm$. Il est également possible de produire un rayonnement harmonique XUV de forte ellipticité en mélangeant deux champs à $800nm$ et $400nm$, polarisés circulairement et de sens de rotation opposés (\mycite{KfirNP2015, FleischerNP2014}). Ce dispositif permet la production d'un rayonnement XUV avec une ellipticité de l'ordre de $90\%$. 
%Enfin, dans un dispositif similaire, \mycite{lambertNC2015} montre qu'en mélangeant deux champs à $800nm$ et $400nm$ polarisés linéairement et orthogonalement entre eux, il est possible de produire des harmoniques ayant une ellipticité XUV de $50\%$ en moyenne sur l'ensemble du spectre.
%
%Cependant, comme nous le verrons, la caractérisation de l'état de polarisation des harmoniques produites par ces différentes méthodes est incomplète. La caractérisation complète de l'état de polarisation est le sujet du chapitre 4. 
%
%\subsubsection{Plan de la thèse}
%
%Les travaux présentés dans ce manuscrit ont commencé en octobre 2012, sous la direction de Pascal Salières. Ils comprennent deux études principales. La première étude se place dans le cadre de la spectroscopie harmonique: dans ce cadre, la caractérisation complète et la plus précise possible de l'état de polarisation des harmoniques est apparue comme une nécessité et une source précieuse d'information, encore peu explorée expérimentalement. Ces travaux font suite à une première campagne, ayant eu lieu en 2012, n'ayant pas donné de résultats définitifs, mais ayant permis de résoudre une série de problèmes expérimentaux. La deuxième étude est associée aux études résolues en temps à l'échelle attoseconde. Elle consiste en la caractérisation complète, résolue spectralement en amplitude et en phase, de la photoémission électronique au voisinage d'une résonance d'autoionisation dans des atomes en phase gazeuse.
%
%
%\begin{itemize}
%\item Dans le premier chapitre, nous rappelerons tout d'abord les outils théoriques permettant de décrire la GHOE. Ensuite, nous présenterons les outils expérimentaux utilisés au cours de cette thèse. 
%\item Dans le second chapitre, le plus technique de ce manuscrit, nous présenterons les outils nécessaires à la caractérisation de l'état de polarisation du rayonnement harmonique. Nous insisterons sur les deux techniques utilisées: la polarimétrie optique qui, dans notre cas, n'offre qu'une description incomplète de l'état de polarisation des harmoniques, puis la polarimétrie moléculaire, technique basée sur le dichroisme circulaire dans les distributions angulaires de photoionisation, et permettant une caractérisation complète de la polarisation. 
%\item Le troisième chapitre traite d'une technique originale de modulation de l'ellipticité du champ fondamental permettant la production d'harmoniques polarisées elliptiquement. Cette technique sera appliquée à la mesure de phase de la composante majeure de l'ellipse harmonique en fonction de l'ellipticité du champ infrarouge. 
%\item Dans le chapitre 4, le plus important de ce manuscrit, nous réexplorons les méthodes de production d'harmoniques polarisées elliptiquement, notamment celles développées par \mycite{ZhouPRL2009}, \mycite{ferreNC2015} et \mycite{FleischerNP2014}. Pour chaque méthode, nous présenterons les caractérisations de l'état de polarisation obtenues, respectivement, par polarimétrie optique et par polarimétrie moléculaire, technique mise en oeuvre dans la collaboration avec l'équipe de Danielle Dowek de l'ISMO. Parmi les informations nouvelles apportées par cette dernière technique, un taux de dépolarisation significatif, qui n'avait pas été caractérisé jusqu'à présent, est mesuré pour les différentes méthodes de génération. Son origine est discutée dans le cas de l'émission harmonique dans $N_2$, à partir des simulations basées sur la QRS (Quantuum ReScattering theory).
%\item Le chapitre 5 traite de la caractérisation en amplitude et phase spectrale, de la photoémission au voisinage de la résonance d'autoionisation $2s2p$ de l'hélium, utilisant la technique RABBIT (\mycite{PaulScience2001}). Nous montrerons que cette caractérisation spectrale donne accès à une information beaucoup plus précise sur le système. Elle permet de reconstruire le paquet d'ondes électronique dans le domaine spectral et donc dans le domaine temporel, permettant d'accéder à la dynamique électronique dans l'état résonant. Des simulations effectuées par le groupe de Fernando Martin de la "Universidad autonoma de Madrid", basées d'une part sur des calculs ab initio et d'autre part sur une théorie perturbative au second ordre, seront présentées. 
%\end{itemize}
%
%Enfin nous conclurons ce manuscrit en rappelant les principaux résultats présentés et en évoquant des perspectives de continuation et de développement de ce travail.
%
%
%
%
%
%
%
%
%
%
%
%
