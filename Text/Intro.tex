\chapternonumtoc{Introduction}
Comme l'énergie ou la quantité de mouvement, le moment angulaire joue un rôle essentiel dans l'étude de dynamiques d'objets en interaction, qu'ils soient matériels ou un rayonnement. Bien avant le nom et la formulation qu'on lui connaît aujourd'hui, le concept de moment angulaire a été discuté lors de l'étude de la rotation inertielle d'un corps et du mouvement de révolution des planètes. Par exemple, Platon discute dans \textit{La République} du mouvement d'une toupie et de l'apparente contradiction d'un objet restant en un même point tout en étant en mouvement. Le concept de moment angulaire est suggéré par Isaac Newton dans le \textit{Principia} : \textit{"A top, whose parts, by their cohesion, are perpetually drawn aside from rectilinear motions, does not cease its rotation otherwise than it is retarded by the air"}. Il y discute également du cas des planètes, qui semblent garder un mouvement circulaire sur de très longues durées. En 1744, D. Bernoulli utilisa le terme de \textit{momenti motus rotatorii}, "moment de mouvement de rotation", qui est peut-être la première conception du moment angulaire moderne. La notation vectorielle utilisée aujourd'hui est ensuite définie par \mycite{Rankine} pour un objet matériel. Le moment angulaire $\bm{J}$ d'un objet matériel ponctuel s'écrit :
\begin{equation}
\bm{J} = \bm{r}\times\bm{p},
\label{eq:am_matiere}
\end{equation}
où $\bm{r}$ et $\bm{p}$ sont respectivement sa position et sa quantité de mouvement.

\subsubsection{\'Energie, quantité de mouvement et moment angulaire de la lumière}
\`A la fin du \textsc{xix}\ieme ~siècle, les travaux de J.C. Maxwell démontrent que le champ électromagnétique se propage dans l'espace sous la forme d'une onde. En s'appuyant sur ces résultats, J.H. Poynting mis en évidence en 1905 que, de même que la matière, la lumière porte de l'énergie et une quantité de mouvement. C'est également lui qui en 1909 réalisa qu'une onde électromagnétique polarisée circulairement porte du moment angulaire selon sa direction de propagation. Il prédit qu'une transformation quelconque de l'état de polarisation, e.g. de linéaire à circulaire, doit s'accompagner d'un échange de moment angulaire entre la matière et la lumière. \\
\`A cette même époque, M. Planck obtient un résultat fondateur de la physique quantique moderne : en étudiant la radiation du corps noir, il formule l'hypothèse selon laquelle tout système qui absorbe ou émet de la lumière à la pulsation $\omega$ a une énergie multiple de $E = \hbar\omega$ \mycite{planck1989}. Motivé par ces travaux, A. Einstein considère la possibilité que ces quanta d'énergie électromagnétique - aujourd'hui appelés photons - aient une réalité physique. Cela lui permet d'expliquer l'effet photoélectrique \mycite{einstein1905}, avant de démontrer qu'un photon doit également porter une quantité de mouvement \mycite{einstein1909}. C'est ce que les expériences de Millikan et Compton démontrèrent quelques années plus tard. 

La quantification du moment angulaire fut quant à elle d'abord obtenue pour les électrons. En 1913, N. Bohr cherche à décrire la structure des atomes, ce qui l'amène à postuler que les électrons gravitent autour du noyau à la manière d'une planète. Il postule de surcroît que le moment angulaire associé à ces orbites est quantifié en unités de $\hbar$ \mycite{Bohr1913}. \`A ce moment angulaire qui sera par la suite nommé \textit{orbital}, s'ajoute une composante intrinsèque mise en évidence par l'expérience de Stern et Gerlach \mycite{Gerlach1922}. Cela amène \mycite{uhlenbeck1925} à définir la quantité qui sera ensuite nommée \textit{spin} de l'électron par W. Pauli. Suite à ces travaux, la théorie de F. Hund et R. S. Mulliken permet de décrire la structure atomique et moléculaire à partir de 4 nombres quantiques \mycite{Hund1926,Mulliken1927}. Parmis ces nombres on trouve le nombre quantique de spin $s$ et le nombre quantique azimutal $\ell$, respectivement associés aux opérateurs de moment angulaire de spin $\hat{S}$ et orbital $\hat{L}$. Ces opérateurs se somment pour donner $\hat{J}$, le moment angulaire total d'un atome ou d'une molécule.

Le cas du moment angulaire du photon est discuté pour la première fois par \mycite{Bose1924}, qui suggère l'existence d'un moment angulaire intrinsèque multiple de $\hbar$, ce qui est finalement démontré expérimentalement par \mycite{Raman1931}. De manière intéressante, ces auteurs font immédiatement le lien entre spin du photon - une quantité microscopique valant $\pm 1$ - et la polarisation de l'onde électromagnétique, grandeur macroscopique. \`A la manière de \mycite{dirac1981}\footnote{La première édition de cette ouvrage est publiée en 1930 par \textit{Cambridge University Press.}}, ils décrivent une polarisation linéaire non pas comme composée de particules sans spin, mais de particules ayant une probabilité égale d'avoir un spin positif ou négatif. De la même manière, une polarisation elliptique est vue comme une probabilité inégale d'avoir un spin d'un des deux signes.   

La première observation du moment angulaire fournit à la matière par le spin du photon est due à \mycite{BethPR1936}, qui mesura le couple exercé par la lumière sur une lame de verre biréfringente. Cette expérience utilise un dispositif similaire à celui proposé par Poynting en 1909, tout en donnant le résultat attendu par la mécanique quantique : l'onde électromagnétique porte un moment angulaire de $\pm\hbar$ par photon, appelé \textit{moment angulaire de spin} (MAS).

Nous avons mentionné que pour un atome ou une molécule, le moment angulaire total se décompose comme la somme de deux composantes : $\hat{J} = \hat{L}+\hat{S}$. Longtemps oublié ou ignoré, le sujet du moment angulaire orbital (MAO) de la lumière est abordé pour la première fois dans le travail de \mycite{Landau1982a}. En 1992, Les Allen et collaborateurs mettent en évidence son existence dans certains faisceaux lumineux \mycite{AllenPRA1992}. Leur travail montre qu'il est relié à la structure spatiale transverse de la lumière, par opposition au MAS lui relié à l'état de polarisation de l'onde. Les faisceaux lumineux utilisés par \mycite{AllenPRA1992} sont les modes de Laguerre-Gauss (LG). Ils présentent une phase transverse hélicoïdale, qui induit une singularité de phase et donc un zéro d'intensité en leur centre. La contribution majeure de \mycite{AllenPRA1992} est de relier cette singularités de phase à la présence de moment angulaire orbital. De plus, les faisceaux de LG utilisés ici constituent une famille de modes du champ électro-magnétique, paramétrée par deux indices couramment notés $(\ell,p)$. Leur profil et leur singularité de phase est donc robuste à la propagation, ce qui donne un moment angulaire orbital identique en tout point. Cette propriété s'applique à tout faisceau ayant une phase hélicoïdale, mais les faisceaux de LG sont les plus couramment utilisés car relativement facile à produire en laboratoire. 

Comme pour la matière, le MAS des photons associés à une onde polarisée circulairement a une projection de $\pm \hbar$. Nous verrons dans la partie \ref{PA:LightAM} que le MAO d'une onde électromagnétique s'écrit quant à lui :
\begin{equation}
\bm{L} = \int_V \bm{r}\times\bm{\Pi}\;\rmd V,
\end{equation}
où $\bm{\Pi}$ est le vecteur de Poynting, quantité de mouvement du champ. On obtient donc une forme très similaire à l'équation \ref{eq:am_matiere}, moment angulaire de la matière. En évaluant sa composante selon l'axe de propagation pour un mode de LG d'indice $\ell$ et en divisant le résultat obtenu par le nombre de photons transportés, on obtient $\bm{L}=\ell\hbar$ par photon. Il est donc très tentant de penser que le MAO, de même que le MAS, est une propriété propre aux photons composant le champ, au sens de l'électrodynamique quantique. La séparation du moment angulaire du photon en deux composantes est en fait loin d'être immédiate \mycite{VanEnk1994} et fait toujours l'objet de nombreuses discussions \mycite{LeaderLorce2014, BarnettJO2016}. Comme nous le mentionnerons à la partie \ref{PA:LightAM}, la séparation est toutefois possible en se plaçant dans les conditions de l'optique paraxiale, et sa validité est démontrée quotidiennement par les nombreuses utilisations des deux types de moment angulaire.

\begin{figure}[!ht]
\centering
\def\svgwidth{\columnwidth}
\import{Figures/Intro/}{momentangulaire_intro.pdf_tex}
\justify{\textsc{Figure} - La lumière peut porter deux types de moment angulaire : un moment angulaire de spin, associé à son état de polarisation, et un moment angulaire orbital, associé à un profil de phase transverse hélicoïdal. On représente à droite une onde plane polarisée circulairement et à gauche un faisceau de Laguerre-Gauss polarisé linéairement.}
\label{Fig:Intro_MA}
\end{figure}

\subsubsection{Utilisations du moment angulaire de la lumière}
L'optique quantique offre un terrain extrêmement fertile d'applications et d'expérimentations mettant en jeu le moment angulaire de la lumière. Depuis le début des années 1980, de nombreux états du rayonnement ont été découverts, tels que les états à un photon ou les paires de photons intriqués. La grandeur la plus couramment manipulée dans ces expériences est la polarisation, ou moment angulaire de spin, de photons uniques \mycite{AspectPRL1982}. En plus de la compréhension de nombreux phénomènes quantiques, la manipulation du MAS a permis l'émergence d'une discipline nouvelle : l'information quantique. Plus récemment, \mycite{MairNature2001} démontrèrent la possibilité d'intriquer des photons grâce à leur moment angulaire orbital, signe que le MAO a un sens au niveau quantique. Ces résultats sont d'importance considérable pour l'information quantique \mycite{LeachScience2010,BorgesPRA2010} : le MAO donne accès a un nombre infini d'états de $\ell$ différents, à comparer aux deux états de MAS possibles, laissant entrevoir des possibilités de multiplexage colossal d'information.

Les propriétés macroscopiques de faisceaux portant du MAO ont également trouvé un grand nombre d'applications. Par exemple, ce moment angulaire peut être transféré à des micro-particules, qui sont alors mises en rotation par la lumière. Ces faisceaux appelés "pinces optiques" ont joué un rôle révolutionnaire dans des domaines allant de la physique atomique, où elles servent à piéger et refroidir des atomes neutres, à la biologie, où elles servent à manipuler bactéries, chromosomes et virus \mycite{AshkinPNAS1997}. La singularité de phase et le zéro d'intensité au centre d'un faisceau de LG ont quant à eux permis de développer de nouvelles microscopies sub-longueur d'onde, telles que les techniques de contraste de phase \mycite{FurhapterOL2005} ou de déplétion par émission stimulée \mycite{HellOL1994}. Enfin, on note un intérêt pour le MAO de la lumière assez récent en astronomie, où il peut être la signature de trous noirs en rotation \mycite{TamburiniNP2011}.

Quant aux faisceaux portant du MAS, leur signature macroscopique est une polarisation circulaire droite ou gauche. Ces états de polarisations furent mis en évidence par les travaux successifs de Malus, Arago, Biot puis Pasteur. Ils sont historiquement liés au concept de \textit{chiralité} de la matière \mycite{Ruchon2005}. Dans le cas de molécules chirales, qui existent sous plusieurs formes appelées énantiomères, images miroirs l'une de l'autre mais non superposables, la lumière polarisée circulairement est une des rares façons de distinguer ces formes entre elles. Les molécules chirales jouant un rôle essentiel en biologie, la lumière polarisée circulairement est utilisée en permanence dans les industries alimentaires et pharmaceutiques ainsi qu'en spectroscopie moléculaire. Par exemple, une des expériences les plus communément réalisées est la mesure de dichroïsme circulaire : une onde polarisée circulairement droite ou gauche est absorbée différemment par l'un ou l'autre des énantiomères, ce qui permet de les distinguer. Le spectre de dichroïsme circulaire est spécifique d’asymétries moléculaires et macromoléculaires, fournissant un outil d'analyse quantitative précieux.

\subsubsection{Vers de plus faibles longueurs d'onde, de plus courtes durées}
L'intégralité des références citées jusqu'ici ont un point commun : elles utilisent des faisceaux de longueur d'onde visible ou proche-infrarouge. En général, il s'agit de la lumière fournie par un laser continu ou impulsionnel si les mesures sont résolues en temps ou si on s'intéresse à des phénomènes de champ fort. Pourtant, une très grande partie de ces applications gagneraient à utiliser une longueur d'onde plus faible. Par exemple, pour la manipulation de particules et pour la microscopie, elle augmenterait directement la résolution spatiale. En spectroscopie moléculaire, elle permettrait d'accéder au régime d'ionisation à un photon. Un exemple de rupture scientifique a été donné par la spectroscopie de molécules chirales : au début des années 2000, on assista au développement de sources synchrotrons produisant de la lumière polarisée circulairement dans le domaine extrême ultraviolet (XUV) \mycite{Nahon2001,nannarone2004}. Ces avancées permirent d'étudier la photoionisation de molécules chirales par une onde polarisée circulairement, mettant en évidence un nouveau phénomène appelé dichroïsme circulaire de photoélectron (PECD) \mycite{GarciaJCP2003}. Ce phénomène présente une sensibilité dépassant celle du dichroïsme circulaire habituel de plusieurs ordres de grandeurs. Bien que démontré dans le régime multi-photonique par la suite, le régime d'ionisation à un photon reste le plus sensible à la structure moléculaire\footnote{On trouvera dans \mycite{BeaulieuNJP2016}, article \ref{pap:BeaulieuNJP2016} inclus à la fin de cette thèse, une comparaison complète de mesures de PECD de la même molécule dans de nombreux régimes d'ionisations différents.}.

Le développement de sources XUV portant un moment angulaire semble donc prometteur. Quelques lignes synchrotron répondent actuellement à la demande de lumière XUV portant un MAS. Bien qu'excellentes en termes de robustesse, équipement et reproductibilité, elles pourraient être utilement complétées par des sources :
\begin{itemize}
\renewcommand{\labelitemi}{$\bullet$}
\setlength\itemsep{1em}
\item Impulsionnelles, dans le régime femto ($10^{-15}$ s) ou attoseconde ($10^{-18}$ s),
\item Portant au choix un MAS ou un MAO,
\item Plus légères qu'un synchrotron.
\end{itemize}
Le développement de telles sources a constitué l'objectif principal de mes travaux.

L'avènement de lasers de plus en plus puissants permet aujourd'hui, pour générer des longueurs d'ondes plus faibles, d'utiliser une large palette d'effets \textit{non-linéaires}. En général, des lois de conservation de l'énergie, de la quantité de mouvement et du moment angulaire déterminent les propriétés du faisceau créé. Ainsi, on peut générer la seconde harmonique d'un faisceau à 800 nm pour diviser sa longueur d'onde par deux et, de manière remarquable, le moment angulaire du fondamental est dans ce cas transféré au faisceau à 400 nm \mycite{CourtialPRA1997}. Suivant cette logique, un candidat de choix pour répondre à nos besoins est la génération d'harmonique d'ordre élevé (GHOE). Il s'agit d'un processus extrêmement non-linéaire qui permet de produire un grand nombre d'harmoniques du fondamental, qui atteignent rapidement des longueurs d'ondes extrêmement courtes. Le rayonnement émis est de plus prodigieusement court temporellement. Dans ce processus, un laser infrarouge impulsionnel et énergétique est focalisé dans un jet de gaz atomique ou moléculaire. Si le milieu est centrosymétrique et si l'impulsion présente plusieurs cycles optiques, on assiste à l'émission d'un rayonnement cohérent composé des harmoniques impaires de la fréquence du laser de génération. De manière remarquable, l'intensité de ces harmoniques ne suit pas un comportement perturbatif. Au contraire, leur intensité est quasiment constante sur une large gamme spectrale. Avec les lasers les plus courants aujourd'hui, le spectre du rayonnement émis se situe typiquement dans la gamme $\sim 10-100$ eV, appelée domaine extrême ultraviolet (XUV). Ce phénomène a été observé pour la première fois par \mycite{FerrayJPB1988} et \mycite{mcphersonJOSAB1987}. Assez rapidement, \mycite{farkas1992} proposent d'utiliser ce spectre très large pour synthétiser une impulsion très brève. Ces grandeurs sont en effet reliées par la relation de Cauchy-Schwarz : plus le spectre est large, plus l'impulsion sera brève. Il a cependant fallu attendre 2001 pour observer ces impulsions ultracourtes, à cause de la difficulté de mesurer la phase spectrale de l'émission. \mycite{PaulS2001,HentschelN2001} parvinrent à mesurer cette phase grâce à la technique RABBIT, démontrant la génération d'impulsions ayant une durée de l'ordre de $\SI{200e-18}{s} = 200$ attosecondes. 

Pour généraliser le principe du transfert de moment angulaire dans la génération de seconde harmonique (SHG) à la GHOE, il reste un sujet à élucider : la conservation du moment angulaire dans le processus de GHOE. \`A la différence de la SHG, la GHOE est un phénomène non-perturbatif, et est de nature physique très différente. Si la conservation de l'énergie ou de la quantité de mouvement y sont observées de façon routinière, celle du MAO et du MAS le sont beaucoup moins. Il nous faudra d'abord étudier l'efficacité du processus, si le laser de génération porte un des deux moments angulaires. Si des harmoniques sont effectivement générées, nous chercherons à caractériser le moment angulaire porté par chacune. Bien sûr, le moment angulaire doit être conservé de manière globale, mais est-il directement transféré au rayonnement XUV, ou bien en partie absorbé par le milieu gazeux de génération ? La GHOE est souvent décrite comme un processus paramétrique où un photon de l'harmonique d'ordre $q$ est issu de l'absorption de $q$ photons du laser de génération. Nous pouvons tester cette représentation : si le moment angulaire suit cette loi, chaque harmonique devrait porter $q$ fois le moment angulaire du fondamental. Une des questions suivantes est celle de l'impulsion attoseconde résultante : par exemple, si chaque harmonique porte un moment angulaire orbital, elle a un profil de phase particulier. Quel est alors le profil spatial et la durée de l'impulsion composée de la totalité des harmoniques émises ? 

\subsubsection{Objectifs}
Dans cette thèse, nous étudierons la GHOE à partir d'un laser impulsionnel infrarouge dont on contrôle le moment angulaire. On cherchera à démontrer la génération de rayonnement XUV ultra-court de MAS ou de MAO bien contrôlés, qui constituerait une source de lumière unique. Dans la \hyperref[PA:GHOE]{première} partie, nous introduisons les bases théoriques et expérimentales de la génération d'harmonique d'ordre élevé dans le cas le plus usuel. Nous présentons un modèle assez simple rendant compte d'une grande partie des caractéristiques du rayonnement. Ensuite, nous décrivons la mise en œuvre expérimentale de la GHOE et montrons les résultats typiquement obtenus.

Dans la \hyperref[PA:LightAM]{deuxième} partie nous proposons une étude générale, théorique, du moment angulaire de la lumière. Nous commençons par une description classique du moment angulaire d'un objet, puis de la lumière. Nous étudions ensuite l'équation d'onde, ce qui nous permet d'obtenir la forme de certaines familles de modes du champ, et en particulier celle des modes de Laguerre-Gauss mentionnés plus haut. Nous adoptons alors une description quantique, faisant apparaître naturellement la conservation des différentes propriétés de la lumière, ainsi que la séparation entre MAS et MAO. Enfin, nous empruntons le formalisme de l'électrodynamique quantique pour définir ces deux quantités pour un photon. Nous discutons au passage de la séparabilité des deux composantes de moment angulaire. Ce formalisme est finalement utilisé pour trouver des modes du champ ayant un moment angulaire bien défini, puis pour étudier l'interaction entre une onde portant du moment angulaire avec un atome.

Avec ces outils en main, nous étudions expérimentalement la GHOE à partir d'un faisceau de Laguerre-Gauss dans la \hyperref[PA:OAM_HHG]{partie 3}. Après avoir expliqué comment contrôler le MAO dans l'infrarouge, nous démontrons la possibilité de générer un rayonnement harmonique. Nous analysons ensuite ses propriétés spatiales, qui à l'aide de calculs analytiques et de simulations numériques, nous renseignent sur la conservation du MAO dans le processus. Nous obtenons ainsi le MAO, $\ell$, porté par chaque harmonique émise. Enfin, nous réalisons des mesures de phase spectrale, nous permettant d'étudier la structure spatio-temporelle du train d'impulsions attosecondes généré. Pour terminer nous étudions le rôle du second indice des modes de LG, $p$, et montrons qu'on peut également agir sur sa valeur.

Dans la \hyperref[PA:Spin_HHG]{quatrième} et dernière partie, nous nous intéressons à la composante de spin du moment angulaire de la lumière. Nous étudions le comportement de la GHOE si le laser de génération est polarisé circulairement et montrons en particulier que l'efficacité du processus est nulle dans ce cas. Nous répondons à cette difficulté en utilisant une résonance du gaz de génération, ce qui nous permet de générer un rayonnement XUV intense et fortement polarisé. Cela nous permet de réaliser des mesures auparavant réservées aux sources synchrotrons mentionnées plus haut. Nous choisissons d'étudier l'interaction du rayonnement harmonique avec des molécules chirales, et plus précisément de mesurer leur dichroïsme circulaire de photoélectron (PECD). Après avoir présenté les résultats de ces expériences, nous les comparons à ceux obtenus sur une source synchrotron dans des conditions similaires. Pour terminer, nous détaillons comment ce dichroïsme peut également être utilisé, non pas pour caractériser la molécule grâce à la lumière, mais pour mesurer l'état de polarisation de la lumière grâce à la molécule.

Enfin, nous présentons une conclusion générale de ce manuscrit, et développons les perspectives ouvertes par ce travail de thèse. Nous discutons des applications envisagées pour ces impulsions XUV au moment angulaire contrôlé, dont certaines sont déjà à l'étude, et du futur des expériences de PECD présentées ici, notamment de la possibilité de leur ajouter une résolution temporelle femto ou attoseconde. 
