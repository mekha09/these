
%\begin{align*}
%\bm{L}&=\epsilon_0\int{\rmd\bm{r}\sum_{a=x,y,z} \bm{E^a_{\bot}}(\bm{r}\times\bm{\nabla})\bm{A^a_{\bot}}}\\
%\hat{\bm{L}}&=\epsilon_0 \sum_{\beta,\beta '} \rmi \omega_{\beta}\mathcal{A}_{\beta}\mathcal{A}_{\beta '}\int{\rmd\bm{r}
%[\hat{a}_{\beta}\bm{F}_{\beta}-\hat{a}^{\dag}_{\beta}\bm{F}^*_{\beta}](\bm{r}\times\bm{\nabla})[\hat{a}_{\beta '}\bm{F}_{\beta '}+\hat{a}^{\dag}_{\beta '}\bm{F}^*_{\beta '}}]\\
%&=\rmi\epsilon_0 \frac{\hbar}{2\epsilon_0} \sum_{\beta,\beta '} \sqrt{\frac{\omega_{\beta}}{\omega_{\beta '}}} (\hat{a}^{\dag}_{\beta}\hat{a}_{\beta '}+\hat{a}_{\beta '}\hat{a}^{\dag}_{\beta}) \int{\rmd\bm{r}\bm{F}^*_{\beta}(\bm{r}\times\bm{\nabla})\bm{F}_{\beta '}}\\
%&= \frac{1}{2} \sum_{\beta,\beta '}\sqrt{\frac{\omega_{\beta}}{\omega_{\beta '}}}(\hat{a}^{\dag}_{\beta}\hat{a}_{\beta '}+\hat{a}_{\beta '}\hat{a}^{\dag}_{\beta})\bra{\bm{F}_{\beta}}\hat{\mathcal{L}}\ket{\bm{F}_{\beta '}},
%\end{align*}
%o� on a not� $\hat{\mathcal{L}} = -\rmi\hbar(\bm{r}\times\bm{\nabla})$ l'op�rateur de la m�canique quantique pour le moment angulaire orbital. On effectue le m�me genre de calcul pour le moment angulaire de spin :
%\begin{align}
%\bm{S}&=\epsilon_0\int{\rmd\bm{r}\bm{E_{\bot}}\times\bm{A_{\bot}}}
%&= \frac{1}{2} \sum_{\beta,\beta '}\sqrt{\frac{\omega_{\beta}}{\omega_{\beta '}}}(\hat{a}^{\dag}_{\beta}\hat{a}_{\beta '}+\hat{a}_{\beta '}\hat{a}^{\dag}_{\beta})\bra{\bm{F}_{\beta}}\hat{\mathcal{S}}\ket{\bm{F}_{\beta '}},
%\label{eqS_QED}
%\end{align}
 %avec cette fois $\hat{\mathcal{S}}=-\rmi\hbar\epsilon_{ijk}$, o� $\epsilon_{ijk}$ est le symbole de Levi-Civita, est l'op�rateur de spin d'une particule de spin 1.
