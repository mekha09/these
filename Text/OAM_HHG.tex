\chapter{Le moment angulaire orbital dans la génération d'harmoniques d'ordre élevé}
\label{CH:OAM_HHG}
%
\section{Revue des utilisations pratiques des modes de Laguerre-Gauss}
\subsection{Le domaine visible et infrarouge}
\subsection{De plus courtes longueurs d'ondes : perspectives d'applications dans l'extrême ultra-violet}
\subsection{Des durées ultra-brèves : utilisations possibles d'impulsions attosecondes portant du moment angulaire orbital}

\section{Génération d'harmoniques d'ordre élevé à partir de modes de Laguerre-Gauss}

Ayant passé en revue les utilisations envisagées de faisceaux de Laguerre-Gauss de durées ultra-brève et dans le domaine de l'extrême ultra-violet (XUV), nous décrirons ici comment les générer de manière expérimentale. Nous commencerons par détailler notre dispositif de génération d'harmoniques d'ordre élevé dans le cas habituel d'un mode laser Gaussien, puis expliquerons comment passer au cas Laguerre-Gaussien avant de donner les résultats obtenus.

\subsection{Le cas Gaussien : Aspects expérimentaux de la génération d'harmoniques d'ordre élevé}
\subsubsection{Système laser}
Toutes les expériences présentées dans ce chapitre ont été réalisées sur le laser LUCA (Laser Ultra-Court Accordable) du LIDYL au CEA Saclay. Il délivre des impulsions ayant une enveloppe temporelle gaussienne de largeur à mi-hauteur $\tau = 50$ fs et une enveloppe spatiale gaussien de largeur à mi-hauteur $w_0 = 15$ mm à $\frac{1}{e^2}$. La longueur d'onde utilisée est 800 nm, et le taux de répétition est de 20 Hz. Ce système dispose d'une fibre utilisée pour filtrer spatialement le faisceau, ce qui garantit un profil très proche d'un mode gaussien pur \mycite{MahieuAPB2015}. Le prix à payer est une diminution de l'énergie par impulsion, qui atteint quand même environ 35 mJ après la fibre et le dernier étage de compression.

\subsubsection{Génération d'harmoniques d'ordre élevé}
Nous commençons par mettre en forme le faisceau laser : son diamètre est ajusté à l'aide d'un iris et son énergie est ajustée grâce à un atténuateur constitué d'une lame demi-onde et d'une paire de polariseur croisés. \`{A} la sortie de cet atténuateur, la polarisation du laser est verticale (S). Le faisceau est ensuite focalisé par une lentille dans un jet de gaz délivré par une vanne pulsée à la fréquence du laser par un système piezo-électrique (Attotech). L'utilisation d'une vanne pulsée permet de n'envoyer du gaz que lorsque le faisceau laser est présent, ce qui limite la pression résiduelle dans les chambres à vide. Ainsi, on peut atteindre une pression assez élevé (XXX) dans la région focale sans que l'émission harmonique ne soit réabsorbée par le gaz résiduel. Un autre paramètre important est le diamètre de l'orifice de la vanne (ici, 150 $\mu m$) : en choisissant un diamètre faible, on crée une extension supersonique du gaz ce qui garantit une longueur d'interaction faible avec le laser. On s'approche ainsi des conditions idéales d'un plan d'atomes, ce qui limite l'importance des effets d'accord de phase dans la GHOE. \par
Le choix du gaz dépend de l'expérience réalisée : on peut par exemple utiliser une molécule dont on étudie la réponse, on parle alors de spectroscopie harmonique. Le gaz n'est pas l'objet d'étude dans notre cas, on préférera donc choisir un système simple, facile à se procurer, et ayant une grande section efficace. Le gaz le plus courant est l'Argon, qui est peu coûteux et génère de manière très efficace. Son potentiel d'ionisation est de 15.76 eV, ce qui donne une énergie de coupure assez faible et qui empêche de générer des ordres harmoniques très élevés. Dans les cas où on désire générer des ordres élevés et nombreux, on pourra utiliser d'autres gaz rares comme le Néon ($I_p$ = 21.6 eV) même si la génération sera moins efficace.

Pour notre système, les paramètres nominaux sont :
\begin{itemize}
\item Le diamètre avant focalisation \O{} $\approx$ 10-15 mm,
\item L'énergie par impulsion de l'ordre de E = 1 mJ,
\item La lentille de longueur focale f = 1 m. \\
\end{itemize}
Pour calculer la valeur de l'intensité pic au foyer, on peut calculer le profil du faisceau après focalisation, par exemple par un calcul numérique. Le champ avant la lentille est défini dans les coordonnées cylindriques $(R,\theta)$ par :
\begin{equation*}
E(R,\theta) = \sqrt{I_0} \exp{\left(-\frac{R^2}{w_0^2}\right)}\times\delta(\frac{\mbox{\O}}{2}-R),
\end{equation*}
où $w_0$ est la largeur du faisceau collimaté avant l'iris, \O{}  est le diamètre de l'iris, $\delta$ est la fonction de Heaviside, et $I_0 = \frac{2E\sqrt{\frac{4\log{2}}{\pi}}}{\tau\pi w_0^2}$.
La focalisation d'un faisceau par une lentille mince peut être calculée par une transformée de Fourier (voir \mycite{Goodman}, un des ouvrages de référence pour l'optique de Fourier, et \mycite{Tan} pour des exemples d'implémentations numériques). Il est alors simple de voir l'effet des différents paramètres expérimentaux. Par exemple, on peut faire varier le diamètre de l'iris : la Figure \ref{Fig:IrisScan} montre le profil du faisceau au foyer quand \O{} varie entre 5 et 25 mm. On voit alors que l'intensité pic au foyer évolue entre 0 et 10 $\times 10^{14} \mbox{W/cm}^2$. On se trouve donc parfaitement dans le régime d'intensité nécessaire à la génération d'harmonique : l'intensité est suffisante pour enclencher une ionisation tunnel mais reste assez faible pour ne pas ioniser et dépléter tout le milieu.

\begin{figure}[!ht]
\centering
\def\svgwidth{\columnwidth}
\import{Figures/Iris_Scan/}{Fig_IrisScan.pdf_tex}
\caption{\'{E}volution du foyer lorsqu'on varie la taille de l'iris. De gauche à droite : (1) profil transverse de l'intensité au foyer, (2) intensité pic, (3) taille du waist. Les paramètres sont les suivants : E = 1 mJ, $w_0$ = 15 mm, $\tau$ = 50 fs, $\lambda$ = 792 nm, f = 1 m et \O{} variant de 5 à 25 mm par pas de 1 mm. Le calcul est réalisé sur une grille de 1025x1025 points correspondant à une taille réelle de 5*\O{}.}
\label{Fig:IrisScan}
\end{figure}

Les harmoniques d'ordre élevé du laser infrarouge sont ainsi générées par le gaz situé près du foyer de la lentille. Ce rayonnement XUV est ensuite ré-imagé par un dispositif composé de deux optiques :
\begin{enumerate}
\item Un miroir torique en or de 50 cm de focale. Le miroir travaille à $11.5\degres$ d'incidence rasante ($78.5\degres$ si on défini l'angle par rapport à la normale au miroir), ce qui permet d'avoir une réflectivité importante et plate sur la gamme spectrale considérée (voir Figure \ref{Fig:TorR}).

\begin{figure}[!ht]
\centering
\def\svgwidth{0.6\columnwidth}
\import{Figures/Reflect_Torique/}{torR.pdf_tex}
\caption{Réflectivité calculée du miroir torique en or à un angle d'incidence de $11.5\degres$. (CXRO, \mycite{Henke1993}).}
\label{Fig:TorR}
\end{figure}

Le miroir toroïdal est positionné dans une configuration 2f-2f de sorte à garder un rapport 1:1 entre le foyer de génération et le second foyer. Ré-imager le foyer de génération est utile car on peut focaliser le rayonnement harmonique dans un deuxième dispositif, comme par exemple un spectromètre à temps de vol dans le cas d'une mesure RABBIT (voir ChapitreXX et YY). Un autre avantage est qu'il éloigne la zone de génération, où la pression est élevée, de la zone de détection, qui requiert souvent un vide de qualité pour que les détecteurs fonctionnent.

\item La deuxième optique est une lame de Si$\mbox{O}_{\mbox{2}}$, qui réalise le rôle de filtrage de l'infrarouge de génération. La lame de silice est traitée antireflet pour l'infrarouge grâce à un dépôt de multicouches. La dernière de ces couches est en silice, ce qui combiné à une bonne qualité de surface permet de réfléchir efficacement le rayonnement harmonique. La Figure \ref{Fig:SilR}, tirée de \mycite{MairessePhD} présente la réflectivité de la lame pour le rayonnement harmonique et infrarouge. La réflectivité dans l'XUV est donc supérieure à 50\% jusqu'à l'ordre $\approx 37$, tandis que moins de 10\% de l'infrarouge est réfléchi. Le filtrage de l'infrarouge de génération est souvent crucial : il constitue un bruit de mesure non négligeable, sans compter qu'il peut facilement endommager des optiques ou des détecteurs en aval s'il est focalisé.

\begin{figure}[!ht]
\centering
\def\svgwidth{\columnwidth}
\import{Figures/Reflect_Silice/}{silR.pdf_tex}
\caption{Réflectivité de la lame de silice. \`{A} gauche, transmission et réflectivité à 800 nm en fonction de l'angle d'incidence rasante. Les pointillés repèrent notre angle de $11.5\degres$. \`{A} droite, réflectivité XUV mesurée (cercles) et donnée par le CXRO (lignes) (\mycite{Henke1993}). Figure adaptée de \mycite{MairessePhD}.}
\label{Fig:SilR}
\end{figure}
\end{enumerate}

Nous souhaiterions ensuite pouvoir imager le spectre harmonique, c'est-à-dire séparer les différents ordres harmoniques et imager leurs propriétés spatiales. Un peu après le second foyer, les harmoniques sont dispersées par un réseau à pas variable Hitachi 001-0437 (voir \mycite{KitaAO1983} pour des détails sur son fonctionnement). L'angle de réflexion d'un rayonnement monochromatique de longueur d'onde $\lambda$ est donné par la formule des réseau :
\begin{equation*}
m\lambda=\frac{\sin{\alpha}+\sin{\beta}}{\sigma},
\end{equation*}
où $m$ est l'ordre de diffraction considéré (généralement 1), $\sigma$ le nombre de trait par mètre (1200 traits/mm dans notre cas), $\alpha$ et $\beta$ les angles d'incidence et de réflexion, définis par rapport à la normale au réseau (la documentation donne $\alpha = 87$\degre0s{} pour un fonctionnement optimal).\par
Le réseau de diffraction est cylindrique : il est focalisant dans la dimension horizontale mais est plan dans la direction verticale. Un rayonnement gaussien de faible largeur spectrale $\Delta\lambda$ et de largeur spatiale $w(z)$ formera donc dans le plan focal du réseau une fine ligne verticale de largeur proportionnelle à $\Delta\lambda$ et de hauteur $w(z)$. On image ainsi à la fois les dimensions spectrale et spatiale, si on suppose la symétrie cylindrique. Ce spectre est imagé par des galettes de micro-canaux couplées à un écran de phosphore, lui-même observé par une caméra CCD Basler A102f. 

L'intégralité du dispositif expérimental est représenté sur la Figure \ref{Fig:ExpG}.
\newpage
\vspace{\baselineskip}
\begin{figure}[!ht]
\centering
\def\svgwidth{\columnwidth}
\import{Figures/Setup_G/}{setupG_wbitmap.pdf_tex}
\caption{Dispositif expérimental de génération et détection d'harmoniques d'ordre élevé.}
\label{Fig:ExpG}
\end{figure}

Les figures \ref{Fig:SpectrumGAr} et \ref{Fig:SpectrumGNe} présentent des spectres obtenus avec ce dispositif en utilisant respectivement l'argon et le néon comme gaz de génération. On observe les ordres harmoniques allant de 13 à 29 dans l'argon, et de 13 à 57 dans le néon, la différence d'énergie de coupure étant attendue puisque le néon a un $I_p$ plus élevé. Sur le spectre de l'argon, on observe clairement les deux trajectoires quantiques de la GHOE : une contribution sur l'axe correspond à la trajectoire courte et une plus divergente et moins intense correspond à la trajectoire longue. Dans le cas du néon, les conditions d'accord de phase utilisées favorisent la trajectoire courte. On remarque également que la divergence de la trajectoire courte (resp. longue) augmente (rep. diminue) avec l'ordre harmonique, jusqu'à ce que les deux trajectoires se confondent dans la coupure. Notons finalement la présence sur le spectre du néon de pics satellites autour des harmoniques les plus basses : il s'agit des harmoniques plus élevées diffractées au second ordre par le réseau.
\begin{figure}[!ht]
\centering
\def\svgwidth{\columnwidth}
\import{Figures/Spectrum_G/}{Spectrum_G_Ar.pdf_tex}
\caption{Spectre d'harmoniques d'ordre élevé générées dans l'argon à partir d'un mode laser gaussien.}
\label{Fig:SpectrumGAr}
\end{figure}
\begin{figure}[!ht]
\centering
\def\svgwidth{\columnwidth}
\import{Figures/Spectrum_G/}{Spectrum_G_Ne.pdf_tex}
\caption{Spectre d'harmoniques d'ordre élevé générées dans le néon à partir d'un mode laser gaussien.}
\label{Fig:SpectrumGNe}
\end{figure}

\subsection{Utilisation de modes de Laguerre-Gauss : Mise en œuvre et premiers résultats}
Ayant décrit la génération ``habituelle'' d'harmonique d'ordre élevé, décrivons maintenant l'expérience où le laser générateur possède un mode de Laguerre-Gauss, dont l'expression a été donnée au chapitre précédent (équation \ref{eq:lgmodes}). Les modes de Laguerre-Gauss possèdent un moment angulaire orbital bien défini grâce à leur phase hélicoïdale. Plus précisément, c'est le terme $e^{\rmi\ell\theta}$, également présent dans les modes de Bessel, qui leur confèrent cette propriété. Il sera question de l'index radial $p$ des modes de Laguerre-Gauss plus loin dans cette thèse, mais il n'a pour l'instant pas d'importance pour notre problème. La première question est donc de savoir comment produire un mode de Laguerre-Gauss d'index $(\ell_{IR},0)$ dans l'infrarouge. 

\subsubsection{Génération de modes de Laguerre-Gauss dans le visible et proche infrarouge}
\subsubsubsection{Superposition de modes de Hermite-Gauss}
\label{sec:hg_modes}
Les faisceaux de LG étant des modes du champ électromagnétique, on peut d'abord penser à modifier le laser lui-même pour qu'il lase directement dans le mode désiré. En introduisant des éléments absorbants dans la cavité, il est a priori possible d'interdire la génération d'un mode Gaussien. En pratique, il est assez compliqué de sélectionner un mode de LG. Il est par contre assez simple de sélectionner un des modes de \textit{Hermite-Gauss}, qui sont les solutions de l'équation d'onde en coordonnées cartésiennes. Ces modes sont souvent appelés modes $\mbox{TEM}_{nm}$, pour ``Transverse Electro-Magnetic'', et le mode Gaussien $\mbox{TEM}_{00}$ n'est simplement que le mode d'index le plus bas. Quelques uns de ces modes sont représentés sur la figure \ref{fig:hgmodes}. En insérant simplement un fil vertical (resp. horizontal) dans la cavité laser, on bloque la génération du $\mbox{TEM}_{00}$ et on obtient 

\begin{figure}[!ht]
\centering
\def\svgwidth{\columnwidth}
\import{Figures/Mode_Converter/}{HG_Modes.pdf_tex}
\caption{Modes de Hermite-Gauss pour différentes valeurs de $(n,m)$. De gauche à droite, $(n,m) =$ (0,0), (1,0), (0,1), (2,0), (2,1), (3,3). Le code couleur est le même que celui de la figure \ref{Fig:LGModes} : la couleur donne la phase et la luminosité l'intensité.}
\label{Fig:hgmodes}
\end{figure}






















\section{Conservation du moment angulaire orbital dans la GHOE}
\subsection{Interprétation des résultats observés à partir de calculs analytiques}
\subsection{Simulations numériques de la propagation de modes de Laguerre-Gauss}
\subsection{Calculs SFA : une simulation complète de l'expérience réalisée}

\section{Rôle des trajectoires quantiques dans la GHOE à partir de faisceaux de Laguerre-Gauss}
\subsection{Observation des contributions des différentes trajectoires quantiques à partir des calculs numériques}
\subsection{Observation expérimentale de ces contributions}
\subsection{Interprétation des résultats obtenus : le rôle de l'index radial des modes de Laguerre-Gauss}

\section{Le profil spatio-temporel des impulsions générées : les ``light springs''}
\subsection{Mesure de la phase spectrale de l'impulsion à partir de la technique RABBIT}
\subsection{Reconstruction du profil spatio-temporel de l'émission}

\section{Contrôle complet du moment orbital angulaire de l'émission dans un schéma à deux faisceaux}
\subsection{Lois de conservations dans un schéma à deux faisceaux}
\subsection{Dispositif colinéaire}
\subsection{Dispositif non colinéaire}

\section{Le reste?}
\subsection{Le FEL?}