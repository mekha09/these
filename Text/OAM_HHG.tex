\chapter{Le moment angulaire orbital dans la génération d'harmoniques d'ordre élevé}
\label{CH:OAM_HHG}
%
\section{Revue des utilisations pratiques des modes de Laguerre-Gauss}
\subsection{Le domaine visible et infrarouge}
\subsection{De plus courtes longueurs d'ondes : perspectives d'applications dans l'extrême ultra-violet}
\subsection{Des durées ultra-brèves : utilisations possibles d'impulsions attosecondes portant du moment angulaire orbital}

\section{Génération d'harmoniques d'ordre élevé à partir de modes de Laguerre-Gauss}

Ayant passé en revue les utilisations envisagées de faisceaux de Laguerre-Gauss de durées ultra-brève et dans le domaine de l'extrême ultra-violet (XUV), nous décrirons ici comment les générer de manière expérimentale. Nous commencerons par détailler notre dispositif de génération d'harmoniques d'ordre élevé dans le cas habituel d'un mode laser Gaussien, puis expliquerons comment passer au cas Laguerre-Gaussien avant de donner les résultats obtenus.

\subsection{Le cas Gaussien : Aspects expérimentaux de la génération d'harmoniques d'ordre élevé}
\subsubsection{Système laser}
Toutes les expériences présentées dans ce chapitre ont été réalisées sur le laser LUCA (Laser Ultra-Court Accordable) du LIDYL au CEA Saclay. Il délivre des impulsions ayant une enveloppe temporelle gaussienne de largeur à mi-hauteur 50 fs et une enveloppe spatiale gaussien de largeur à mi-hauteur 32,4 mm à $\frac{1}{e^2}$. La longueur d'onde utilisée est 800 nm, et le taux de répétition est de 20 Hz. Ce système dispose d'une fibre utilisée pour filtrer spatialement le faisceau, ce qui garantit un profil très proche d'un mode gaussien pur \mycite{MahieuAPB2015}. Le prix à payer est une diminution de l'énergie par impulsion, qui atteint quand même environ 35 mJ après la fibre et le dernier étage de compression.

\subsubsection{Génération d'harmoniques d'ordre élevé}
Nous commençons par mettre en forme le faisceau laser : son diamètre est ajusté à l'aide d'un iris et son énergie est ajustée grâce à un atténuateur constitué d'une lame demi-onde et d'une paire de polariseur croisés. \`{A} la sortie de cet atténuateur, la polarisation du laser est verticale (S). Le faisceau est ensuite focalisé par une lentille dans un jet de gaz délivré par une vanne pulsée à la fréquence du laser par un système piezo-électrique (Attotech). L'utilisation d'une vanne pulsée permet de n'envoyer du gaz que lorsque le faisceau laser est présent, ce qui limite la pression résiduelle dans les chambres à vide. Ainsi, on peut atteindre une pression assez élevé (XXX) dans la région focale sans que l'émission harmonique ne soit réabsorbée par le gaz résiduel. Un autre paramètre important est le diamètre de l'orifice de la vanne (ici, 150 $\mu m$) : en choisissant un diamètre faible, on crée une extension supersonique du gaz ce qui garantit une longueur d'interaction faible avec le laser. On s'approche ainsi des conditions idéales d'un plan d'atomes, ce qui limite l'importance des effets d'accord de phase dans la GHOE.
Le choix du gaz dépend de l'expérience réalisée : on peut par exemple utiliser une molécule dont on étudie la réponse, on parle alors de spectroscopie harmonique. Le gaz n'est pas l'objet d'étude dans notre cas, on préférera donc choisir un système simple, facile à se procurer, et ayant une grande section efficace. Le gaz le plus courant est l'Argon, qui est peu coûteux et génère de manière très efficace. Son potentiel d'ionisation est de 15.76 eV, ce qui donne une énergie de coupure assez faible et qui empêche de générer des ordres harmoniques très élevés. Dans les cas où on désire générer des ordres élevés et nombreux, on pourra utiliser d'autres gaz rares comme le Néon ($I_p$ = 21.6 eV) même si la génération sera moins efficace. 
%
Pour notre système, les paramètres nominaux sont :
\begin{itemize}
\item Le diamètre avant focalisation est \O = 10 à 15 mm,
\item L'énergie par impulsion est de l'ordre de E = 1 mJ,
\item La lentille a une longueur focale de f = 1 m.
\end{itemize}
%
%\subsection{Utilisation de modes de Laguerre-Gauss : Mise en œuvre et premiers résultats}
%
%
%
%
%
%
%
%
%
%
%
%
%
%
%
%
%
%
%
%
%
%
%
%
%
%
%\section{Conservation du moment angulaire orbital dans la GHOE}
%\subsection{Interprétation des résultats observés à partir de calculs analytiques}
%\subsection{Simulations numériques de la propagation de modes de Laguerre-Gauss}
%\subsection{Calculs SFA : une simulation complète de l'expérience réalisée}
%
%\section{Rôle des trajectoires quantiques dans la GHOE à partir de faisceaux de Laguerre-Gauss}
%\subsection{Observation des contributions des différentes trajectoires quantiques à partir des calculs numériques}
%\subsection{Observation expérimentale de ces contributions}
%\subsection{Interprétation des résultats obtenus : le rôle de l'index radial des modes de Laguerre-Gauss}
%
%\section{Le profil spatio-temporel des impulsions générées : les ``light springs''}
%\subsection{Mesure de la phase spectrale de l'impulsion à partir de la technique RABBIT}
%\subsection{Reconstruction du profil spatio-temporel de l'émission}
%
%\section{Contrôle complet du moment orbital angulaire de l'émission dans un schéma à deux faisceaux}
%\subsection{Lois de conservations dans un schéma à deux faisceaux}
%\subsection{Dispositif colinéaire}
%\subsection{Dispositif non colinéaire}
%
%\section{Le reste?}
%\subsection{Le FEL?}