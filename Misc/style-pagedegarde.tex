%% Position du NNT
\begin{textblock}{13}(1.15,2.5)
  NNT : \NNT
\end{textblock}


%% Logos en haut de la page
\begin{textblock}{1}(1.15,1)
\includegraphics[height=1.8cm]{Figures/Logos/UPSac.png} %% Logo de Paris Saclay
\label{Logo Paris Saclay}
\end{textblock}

\begin{textblock}{2.47}(12.38,\vpos)
\logoEt %% Logo de votre établissement
\label{Logo Etablissement}
\end{textblock}

\vspace*{1cm}
%% Texte
\color{blue!20!red!45!black} %% Couleur violette du premier paragraphe
  \begin{center}    
    \LARGE\textsc{Thèse de doctorat\\ de\\ l'Université Paris-Saclay} \\
    \LARGE{\textsc{préparée à \\ \PhDworkingplace}} \\ \bigskip
  \color{black} %% Couleur noir du reste du texte
	\bigskip 
	\Large{\textsc{\laboratory}}\\
	\bigskip
    \Large{\textsc{Ecole doctorale n$^{\circ}\ecodocnum$ :}} %% Numéro ED
     \Large{\ecodoctitle}  \\

     \Large{Spécialité : \PhDspeciality} \\%% Spécialité
    \vfill
    \Large{Par} \\ %% Titre de la thèse
    \bigskip
		\bigskip
		   \LARGE{{\textsc{\PhDname}}}\\
   \bigskip
	\bigskip
	\bigskip
    %% Nom du docteur
	\LARGE{\textbf{\textsc{\PhDTitleFR}}}
    \vfill
    \bigskip
		\vspace{3in}
\end{center}
\color{black}
%% Jury

%% Members of the jury
%% If needed, one can add jurymemberG or remove one jury member.
%

\begin{textblock}{13.70}(1.15,11)

\begin{flushleft}
\textbf{Thèse présentée et soutenue à \defenseplace, le \defensedate :} \\
\bigskip
\textbf{Composition du Jury :} \\
\bigskip
\begin{tabularx}{\textwidth}{@{}lll}%{r X l}
    \textsc{\jurygenderC \jurynameC} & \jurygradeC, \juryadressC & \juryroleC \\
    
		\textsc{\jurygenderA \jurynameA}  & \jurygradeA, \juryadressA & \juryroleA \\ 
   
    \textsc{\jurygenderB \jurynameB}   & \jurygradeB, \juryadressB & \juryroleB \\
		
		\textsc{\jurygenderD \jurynameD}  & \jurygradeD, \juryadressD & \juryroleD \\
		
		\textsc{\jurygenderE \jurynameE}   & \jurygradeE, \juryadressE & \juryroleE \\
		
		\textsc{\jurygenderG \jurynameG}   & \jurygradeG, \juryadressG & \juryroleG \\
				
		\textsc{\jurygenderF \jurynameF}   & \jurygradeF, \juryadressF & \juryroleF \\
		%\jurygenderA & \textsc{\jurynameA}\dotfill  & \juryroleA \\ 
   %
    %\jurygenderB & \textsc{\jurynameB}\dotfill  & \juryroleB \\
    %
    %\jurygenderC & \textsc{\jurynameC}\dotfill  & \juryroleC \\
		%
		 %\jurygenderD & \textsc{\jurynameD}\dotfill  & \juryroleD \\
		%
		 %\jurygenderE & \textsc{\jurynameE}\dotfill  & \juryroleE \\
		%
		 %\jurygenderF & \textsc{\jurynameF}\dotfill & \juryroleF \\
    %
    %\jurygenderD & \textsc{\jurynameD}  & \jurygradeD & (\juryroleD) \\
    %%\null & \null & \juryadressD &\\ 
    %%
    %\jurygenderE & \textsc{\jurynameE}  & \jurygradeE & (\juryroleE) \\
    %%\null & \null & \juryadressE &\\ 
    %%
    %\jurygenderF & \textsc{\jurynameF}  & \jurygradeF & (\juryroleF) \\
    %%\null & \null & \juryadressF &\\ 
   %
  \end{tabularx}    
\end{flushleft}
\end{textblock}
