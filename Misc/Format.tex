%%%%%%%%%%%%%%%%%%%%%%%%%%%%%%%%%%%%%%%%%%%%%%%%%%%%%%%%%%%%%%%%%%%%%%%%%%%
% Package loading that allows compilation to pdf either with dvips+ps2pdf or pdflatex.
% For pdf figures must be in format pdf (or jpg). This can be done with epstopdf (%f in (*.eps) do epstopdf %f)
% For pdf via dvi and ps figures must be in format eps.

%\newif\ifpdf\ifx\pdfoutput\undefined\pdffalse\else\pdfoutput=1\pdftrue\fi% Latex mode i.e. \pdfoutput is only defined compiling with pdflatex
%Above row is commented since this declaration is already done in the memoir class
\ifpdf
  \usepackage[pdftex]{graphicx}
  %\usepackage{epstopdf} % This does not work on windows.
  \usepackage[pdftex,dvipsnames,usenames]{color} % This package allows for use of colors
  \DeclareGraphicsExtensions{.pdf,.png}
  \pdfadjustspacing=1  % Use LaTeX like spacing and not pdflatex layout algorithm
  %\usepackage[pdftex]{thumbpdf}
  \usepackage[protrusion={true, compatibility},expansion={true, compatibility}]{microtype} % Margin kerning i.e. prettier margins
\else
  \usepackage{graphicx}
  \usepackage[dvips,dvipsnames,usenames]{color} % This package allows for use of colors
  %\usepackage[ps2pdf]{thumbpdf}
  \DeclareGraphicsExtensions{.eps}
\fi

%R. Géneaux - Fixes the "too many alphabet" errors :
\newcommand\hmmax{0}
\newcommand\bmmax{0}


%--------------------------------------------------------------------------------%
%\usepackage[pagebackref]{hyperref}%A Camper for bak paging the reference
%\usepackage{utopia}					% Utopia font as default roman
\usepackage{mathpazo}				% Zapf Palatino font as default roman. Math in Palatino where possible
%\usepackage{nomencl}         % The nomenclature package can be used to generate and format a
                             % nomenclature using MakeIndex. I use to make a List of Symbols
\usepackage{braket}          % braket: Macros for Dirac bra-ket <|> notation and sets {|}'
\usepackage{array}           % Extends the implementation of array- and tabular-enviroments
%\usepackage[numbers,square,comma,sort&compress]{natbib}
\usepackage[authoryear,square,comma,sort&compress]{natbib}
\usepackage{systeme}
%\usepackage[authoryear,sort,colon,square]{natbib}
\usepackage{ifthen}
\usepackage{calc}
\usepackage{url}
 \usepackage{eurosym,skull}
\usepackage{amsmath,amssymb,graphicx}%,stmaryrd}% Installé par Antoine pour \sslash 
\usepackage{MnSymbol}
\usepackage{picins}
\usepackage{bm}							%gives \bm{xx} command in math mode - more effective way to use bold variables
%\usepackage{color}
%\usepackage{setspace}
%\usepackage{textcomp}        % Defines for example dregrees celcius symbol
%\usepackage{siunitx}         % The SIunitx package provides macros to type numbers and
                             		% units in a consistent way according to SI requirements.
%\usepackage{ulem}            % Allows line through text, \sout{},  under line, \uline{}
                             % wavy under line, \uwave{}, double under line, \uuline{}
                             % text market with slashes, \xout{}
%\normalem                    % Stops ulem of taking over the \emph{} command
\usepackage[utf8]{inputenc}
\usepackage[T1]{fontenc}%A Camper problème d'accents dans l'hyphénation
\usepackage[french]{babel}
\addto\captionsfrench{\def\tablename{\textsc{Tableau}}} %change "`table"' to tableau
\usepackage{icomma} 
\usepackage[table]{xcolor}% http://ctan.org/pkg/xcolor
\DeclareUnicodeCharacter{00A0}{ }
\DeclareUnicodeCharacter{00A0}{ }
%
\ifpdf
  %\usepackage[pagebackref]{hyperref}
	\usepackage[backref=page]{hyperref}
  %\usepackage[pdftex]{hyperref}
\else
  \usepackage[ps2pdf]{hyperref}
  %\usepackage[hypertex]{hyperref}  %% Gives hyperreferenses in the dvi file.
  % \hypersetup{pdffitwindow=false}
\fi
%\usepackage{hypernat} %% Makes the hyperref and natbib packages work together i.e. [1-3] works
%\usepackage{memoir/hypernat} %% Makes the hyperref and natbib packages work together i.e. [1-3] works
\usepackage{memhfixc} %% The memhfixc package provides hyperref related temporary
                      %% fixes and extensions for version v1.3a of the memoir class.

%R. Géneaux
\usepackage{siunitx} %macros for proper typeset SI units
\sisetup{
list-final-separator = { et }
}
\usepackage{titlesec}
\usepackage{import} %The import package allows to use latex+pdf from inkscape when files are in a different folder

\usepackage{transparent} %Handles transparency when importing svg files
\usepackage[all]{xy} %Used for drawings with arrows
\usepackage{ragged2e} %Justify command

\titleformat{\paragraph}
{\normalfont\normalsize\bfseries}{\theparagraph}{1em}{}
\titlespacing*{\paragraph}
{0pt}{3.25ex plus 1ex minus .2ex}{1.5ex plus .2ex}

%RGéneaux - reduce space after subsubsection
\titlespacing{\subsubsection}{0pt}{\parskip}{-5pt}

% Appearance
\hypersetup{
%bookmarks = true,
bookmarksnumbered = true,
breaklinks = true
}

\ifthenelse{\equal{\digitalversion}{y}}
{
  \hypersetup{
  colorlinks = true,
  linkcolor = blue,
  citecolor = red,
	urlcolor = blue!75!black
  }
}
{
  \hypersetup{
  colorlinks = true,
  linkcolor = black,
  citecolor = black,
  filecolor = black,
%  pagecolor = black,
  urlcolor = black
  }
}

\newlength{\mytextwidth}
\newlength{\mymarginwidth}
\setcounter{secnumdepth}{4}

%%%%%%%%%%%%%%%%%%%%%%%%%%%%%%%%%%%%%%%%%%%%%%%%%%%%%%%%%%%%%%%%%%%%%%%%%%%
% Will the thesis be in A4 or B5 format?
\ifthenelse{\equal{\finalsizeB5}{y}}
{
	%%%%%%%%%%%%%%%%%%%%%%%%%%%%%%%%%%%%%%%%%%%%%%%%%%%%%%%%%%%%%%%%%%%%%%%%%%%
	% Page layout
	\setlength{\mytextwidth}{10.1cm}
	\setlength{\mymarginwidth}{4.5cm}
	\trimLmarks
	
	%%%%%%%%%%%%%%%%%%%%%%%%%%%%%%%%%%%%%%%%%%%
	% Scale to B5
	\ifthenelse{\equal{finalsizeB5}{y}}{
	    \ifthenelse{\equal{\makepreprint}{n}}
	        {\setstocksize{250mm}{176mm}
	        \trimNone
	    }{}
	}{}
	%%%%%%%%%%%%%%%%%%%%%%%%%%%%%%%%%%%%%%%%%%%
	
	%%%%%%%%%%%%%%%%%%%%%%%%%%%%%%%%%%%%%%%%%%%
	% Preprint?
	\ifthenelse{\equal{\makepreprint}{y}}
	{
	    \setlrmarginsandblock{2cm}{2.3cm}{*}        % Spine and edge margin
	    \setmarginnotes{0.1cm}{0.1cm}{0.1cm}        % Margin notes: Horizontal distance from text, Width, Vertical separation
	    \OnehalfSpacing
	}
	{
	    \settrimmedsize{250mm}{176mm}{*}          % B5 paper (250 x 176 mm)
	    \settypeblocksize{*}{\mytextwidth}{*}       % Text width
	    \setlrmargins{2cm}{*}{*}                    % Spine margin
	    \setmarginnotes{0.5cm}{\mymarginwidth}{0.5cm}        % Margin notes: Horizontal distance from text, Width, Vertical separation
	}
	\setulmarginsandblock{2.5cm}{2.5cm}{*}      % Top and bottom margins (excluding heading and footer)
	\setheadfoot{1.5cm}{1cm}                    % Header and footer locations
	
	% Center trimmed region on stock
	\setlength{\trimtop}{\stockheight}
	\addtolength{\trimtop}{-\paperheight}
	\setlength{\trimedge}{\stockwidth}
	\addtolength{\trimedge}{-\paperwidth}
	\settrims{0.5\trimtop}{0.5\trimedge}
}
{
	%%%%%%%%%%%%%%%%%%%%%%%%%%%%%%%%%%%%%%%%%%%
	% Preprint?
	\ifthenelse{\equal{\makepreprint}{y}}
	{
	    \setlrmarginsandblock{2cm}{2.3cm}{*}        % Spine and edge margin
	    \setmarginnotes{0.1cm}{0.1cm}{0.1cm}        % Margin notes: Horizontal distance from text, Width, Vertical separation
	    \OnehalfSpacing
	}
	{
			%\semiisopage
			\setlength{\textwidth}{400pt}
			\setlength{\textheight}{684pt}
			\setulmargins{*}{*}{1}
			\setlrmargins{*}{*}{1}
			\setmarginnotes{0.1\foremargin}{0.8\foremargin}{0.1\foremargin}
			%\newlength{\temp}
			%\setlength{\temp}{\paperwidth-\textwidth-\spinemargin-\marginparsep-\marginparwidth}
			%\addtolength{\marginparsep}{\temp/9}
			%\addtolength{\marginparwidth}{3\temp/9}
			%\addtolength{\textheight}{84pt}
			\setlength{\mytextwidth}{\textwidth}
			\setlength{\mymarginwidth}{\marginparwidth}
	}
}	
%%% Fix the layout
\checkandfixthelayout

%%%%%%%%%%%%%%%%%%%%%%%%%%%%%%%%%%%%%%%%%%%%%%%%%%%%%%%%%%%%%%%%%%%%%%%%%%%
% Define useful lengths
\setlength{\headwidth}{\textwidth}
\addtolength{\headwidth}{\marginparsep}
\addtolength{\headwidth}{\marginparwidth}

\newlength{\marginwidth}
\setlength{\marginwidth}{\marginparwidth}
\addtolength{\marginwidth}{\marginparsep}

\setlength{\epigraphwidth}{0.66\textwidth}
\setlength{\epigraphrule}{0pt}
\epigraphtextposition{flushright}

%%%%%%%%%%%%%%%%%%%%%%%%%%%%%%%%%%%%%%%%%%%%%%%%%%%%%%%%%%%%%%%%%%%%%%%%%%%
% Page Style
\makepagestyle{mypagestyle}
\makerunningwidth{mypagestyle}{\headwidth}
\makeheadposition{mypagestyle}{flushright}{flushleft}{flushright}{flushleft}
\makeoddhead{mypagestyle}{}{}{\slshape\small \leftmark}
\makeevenhead{mypagestyle}{\slshape\small \rightmark}{}{}
\makeheadrule{mypagestyle}{\headwidth}{\normalrulethickness}
\makeoddfoot{mypagestyle}{}{}{\slshape\small \thepage}
\makeevenfoot{mypagestyle}{\slshape\small \thepage}{}{}

% Footer for Part and chapter pages
\makerunningwidth{plain}{\headwidth}
\makeheadposition{plain}{flushright}{flushleft}{flushright}{flushleft}
\makeoddfoot{plain}{}{}{\slshape\small \thepage}

% Pagestyle for toc. Headings but no page number
\copypagestyle{mycontentstyle}{mypagestyle}
\makeoddfoot{mycontentstyle}{}{}{}
\makeevenfoot{mycontentstyle}{}{}{}

% What goes in the left and right header
\renewcommand{\chaptermark}[1]{\markboth{\textbf{#1}}{}}
\renewcommand{\sectionmark}[1]{\markright{\thesection\,\ #1}{}}
\renewcommand{\subsectionmark}[1]{\markright{\thesubsection\,\ #1}{}}


%%%%%%%%%%%%%%%%%%%%%%%%%%%%%%%%%%%%%%%%%%%%%%%%%%%%%%%%%%%%%%%%%%%%%%%%%%%
% Part style (i.e. for the page preceeding the papers)
\newcounter{bands}
\newlength{\bandwidth}
\setlength{\bandwidth}{10cm} 
\newlength{\bandheight}
\setlength{\bandheight}{33pt}
\newlength{\bandoverlap}
\setlength{\bandoverlap}{5mm} 
\newlength{\bandrightmargin}
\setlength{\bandrightmargin}{5mm}
\newlength{\fullmarginwidth}
\setlength{\fullmarginwidth}{\paperwidth-\textwidth-\spinemargin}

%% commented - r.géneaux
%\renewcommand{\beforepartskip}{}
%\renewcommand{\printparttitle}[1]{
%\pdfbookmark[-1]{Papers}{Papers}
%\begin{adjustwidth}{}{-\fullmarginwidth-\bandoverlap}
    %\begin{flushright}
    %\colorbox{black}{\textcolor{white}{\normalfont\HUGE\scshape \parbox[t]{\bandwidth+\bandrightmargin+\bandoverlap}{
        %\vspace*{5pt}
        %\begin{flushright}
        %#1 \hspace*{\bandoverlap}\hspace*{\bandrightmargin}
        %\end{flushright}
        %\vspace*{1pt} % was 5
        %\addtocounter{bands}{-1}
        %\vspace{\value{bands}\bandheight}
        %\addtocounter{bands}{1}
    %}}}
    %\end{flushright}
%\end{adjustwidth}
%\thispagestyle{empty}
%}


%%%
%\renewcommand*{\thepart}{\arabic{part}}
%\renewcommand*{\parttitlefont}{\normalfont\large\MakeUppercase}
%\renewcommand*{\partnamefont}{\normalfont\scshape\MakeLowercase}
%\renewcommand*{\partnumfont}{\normalfont\scshape\MakeLowercase}
%
%%Pour "première partie"
%%\renewcommand*{\printpartname}{\partnamefont{\ordinalstring{part}[f] partie}}
%%\renewcommand*{\printpartnum}{}
%
%%Pour "partie 1"
%\renewcommand*{\printpartname}{\partnamefont Partie}
%\renewcommand*{\printpartnum}{\partnumfont\thepart}
%
%\renewcommand{\midpartskip}{\par\parbox{0.5in}{\hrulefill}\par}
%\renewcommand{\beforepartskip}{\vspace*{\fill}}
%\renewcommand{\afterpartskip}{\vspace*{\fill}}

\makeatletter

% define a user command to choose the image
% this command also creates an internal command to insert the image
\newcommand{\partimage}[2][]{\gdef\@partimage{\includegraphics[#1]{#2}}}
\def\@part[#1]#2{%
  \ifnum \c@secnumdepth >-2\relax \refstepcounter{part}%
    \addcontentsline{toc}{part}{\thepart.
        \protect\enspace\protect\noindent#1}%
  \else
    \addcontentsline{toc}{part}{#1}\fi
  \begingroup\centering
  \ifnum \c@secnumdepth >-2\relax
       {\fontsize{\@xviipt}{22}\bfseries \normalfont\scshape\MakeLowercase
         \partname} %\vskip 20\p@ \fi
				\par\parbox{0.5in}{\hrulefill}\par
  \fontsize{\@xxpt}{22}\bfseries \normalfont\scshape\MakeLowercase
      #1\vfil\@partimage\vfil\endgroup \newpage\thispagestyle{empty}}
\makeatother

%% Pour la table des matières
%\renewcommand*{\cftpartname}{Partie}
%\renewcommand*{\cftpartpresnum}{\space}
%\renewcommand*{\cftpartaftersnum}{.}
%\renewcommand*{\cftpartaftersnumb}{\space}
%%

%%%%%%%%%%%%%%%%%%%%%%%%%%%%%%%%%%%%%%%%%%%%%%%%%%%%%%%%%%%%%%%%%%%%%%%%%%%
% Chapter styles
\makeatletter
\makechapterstyle{mychapterstyle}{%
    \renewcommand{\printchaptername}{
        \fixchaptersidebar
        \hrule\vskip\onelineskip
        \begin{adjustwidth}{}{-\marginwidth}
            \raggedleft \normalfont\LARGE\scshape \@chapapp
    }
    \renewcommand{\printchapternum}{
            \normalfont\Huge \thechapter
        \end{adjustwidth}
    }
    \renewcommand{\printchaptertitle}[1]{%
        \begin{adjustwidth}{}{-\marginwidth}
        \raggedright \normalfont\HUGE\scshape ##1\par\nobreak
        \end{adjustwidth}
        \vskip\onelineskip \hrule\vspace*{1.5pt}\hrule
    }
}
\makeatother

\makechapterstyle{papers}{%
    \setlength{\beforechapskip}{-14pt}
    \addtolength{\beforechapskip}{-\bandheight}
    \setlength{\afterchapskip}{5pt}
    \renewcommand{\printchaptertitle}[1]{%
        \begin{adjustwidth}{}{-\fullmarginwidth-\bandoverlap}
            \vspace*{\value{paper}\bandheight}
            \begin{flushright}
            \colorbox{black}{\textcolor{white}{ \normalfont\HUGE\scshape \parbox[t]{\bandwidth+\bandrightmargin+\bandoverlap}{
                \vspace*{3pt} % was 5
                \begin{flushright}
                ##1 \hspace*{\bandoverlap}\hspace*{\bandrightmargin}
                \end{flushright} % was 5
                \vspace*{3pt}
            }}}
            \end{flushright}
        \end{adjustwidth}
        \thispagestyle{empty}
    }
    \newcounter{paper}
    \newcounter{paperpage}

    % Increase indentation in ToC
    \addtocontents{toc}{\protect\addtolength{\cftchapternumwidth}{10pt}}
}


% Macros for adding unnumbered chapters
\newcommand{\chapternonum}[1]{
\cleardoublepage
\pdfbookmark[0]{#1}{#1}
\chapter*{#1}
\markboth{#1}{#1}
}                           % Abstract, etc.

\newcommand{\chapternonumtoc}[1]{
\makeatletter
\def\toclevel@chapter{-1}
\def\toclevel@section{0}
\def\toclevel@subsection{1}
\makeatother

\cleardoublepage
\phantomsection
%\addcontentsline{toc}{chapter}{#1}
\addcontentsline{toc}{part}{#1}
\chapter*{#1}
\markboth{#1}{#1}

\makeatletter
\def\toclevel@chapter{0}
\def\toclevel@section{1}
\def\toclevel@subsection{2}
\makeatother
}                           % Acknowledgements

\newcommand{\chapternonumtitle}[2]{
\chapter*{#1}
\markboth{\textbf{#1}}{#2}
}                           % Papers


%%%%%%%%%%%%%%%%%%%%%%%%%%%%%%%%%%%%%%%%%%%%%%%%
% Paper Handling

\newcommand{\paperjournalref}[3]{
\ifthenelse{\equal{#2}{}}
{#3\ifthenelse{\equal{#1}{}}{}{ \textit{#1}}.}
{\textit{#1} \textbf{#2}, #3.}
}


% Macro for inserting a paper page
\newcommand{\paperpage}[3]{\begin{adjustwidth*}{}{-\marginwidth}
  \centering\noindent\includegraphics[trim=#2, clip=true,
  width=1\textwidth+\marginwidth, height=0.99\textheight,
  keepaspectratio=true]{#1/p_#3}\clearpage\end{adjustwidth*}}

% Macro for adding a paper
\makeatletter
\newcommand{\paper}[9]{\addtocounter{paper}{1}
\cleardoublepage
\phantomsection
\addcontentsline{toc}{chapter}{\protect\numberline{\Roman{paper}}#4}
\begingroup
\def\@currentlabel{\Roman{paper}}
\label{pap:#1}
\endgroup
\chapternonumtitle{Article~\Roman{paper}}{#4}
\paperinfo{#4}{#5}{#6}{#7}{#8}
\addtocontents{lop}{\paperlistentry{#4}{#5}{#6}{#7}{#8}}
\addtocontents{loc}{\papercommententry{#4}{#9}}
\cleardoublepage
\setcounter{paperpage}{0}
%{\setcounter{paperpage}{#3}} % Uncomment to remove paper pages completely
\whiledo{\value{paperpage}<#3}{
\addtocounter{paperpage}{1}
\ifthenelse{\equal{\includepapers}{y}}
{\paperpage{Papers/#1}{#2}{\thepaperpage}}
{\ \clearpage}
}}
\makeatother

% Macro for printing paper info on paper page
\newcommand{\paperinfo}[5]{
\begin{adjustwidth}{}{-\fullmarginwidth+\bandrightmargin}
    \begin{flushright}
    \parbox[t]{\bandwidth}{
        \begin{flushleft}
        \large\textbf{#1}\\
        \vspace{3pt}
        \small#2.\\
        \vspace{2pt}
        \paperjournalref{#3}{#4}{#5}
        \end{flushleft}
    }
    \end{flushright}
\end{adjustwidth}
}

%%%%%%%%%%%%%%%%%%%%%%%%%%%%%%%%%%%%%%%%%%%%%%%%
%Define justified environment
\makeatletter
\newcommand{\justified}{%
  \rightskip\z@skip%
  \leftskip\z@skip}
\makeatother

% List of papers
\newlistof{listofpapers}{lop}{\loptitle}
\renewcommand{\afterloptitle}{\afterchaptertitle \preloptext\par}

% Entry style
\newcommand{\paperlistentry}[5]{%
\protect\addtocounter{bands}{1}
\protect\paperlistitem{\Roman{paper}}{#1}{#2}{#3}{#4}{#5}
}

\newcommand{\paperlistitem}[6]{%
\protect\noindent\parbox{\textwidth}{
\protect\begin{itemize}
\protect\item[#1]{
\protect\begin{flushleft}
\textbf{#2}\\
#3.\\
\protect\noindent\paperjournalref{#4}{#5}{#6}
\protect\end{flushleft}
}
~~\\
\protect\end{itemize}
}}



%%%%%%%%%%%%%%%%%%%%%%%%%%%%%%%%%%%%%%%%%%%%%%%%
% Comments on the papers
\newlistof{commentsonpapers}{loc}{\loctitle}
\renewcommand{\afterloctitle}{\afterchaptertitle \preloctext \par}

% Entry style
\newcommand{\papercommententry}[2]{
\protect\parbox{\textwidth}{
\protect\begin{itemize}
\protect\item[\Roman{paper}]{
\protect\begin{flushleft}
\textbf{#1}\\
\protect\end{flushleft}
#2}
\protect\end{itemize}
}}


%%%%%%%%%%%%%%%%%%%%%%%%%%%%%%%%%%%%%%%%%%%%%%%%
% Enumerations in the text
\renewcommand{\theenumi}{\roman{enumi}}
\renewcommand{\labelenumi}{(\theenumi)}


%%%%%%%%%%%%%%%%%%%%%%%%%%%%%%%%%%%%%%%%%%%%%%%%
% Bibliography style
%\newcommand{\bibfont}{\footnotesize}
%\setlength{\bibsep}{5pt}
%\setbiblabel{#1.\hfill}
\renewcommand{\prebibhook}{\markboth{\bibname}{\bibname}}
\nobibintoc

%A Camper Pages des citations référencées en français dans la biblio
%\newcommand{\mapolicebackref}[1]{
    %\hspace*{\fill} \mbox{\newline\textit {\small #1}}
%}	
%\renewcommand*{\backref}[1]{}
%\renewcommand*{\backrefalt}[4]{
%\ifcase #1 \mapolicebackref{pas de citations}
    %\or \mapolicebackref{Cité page #2}
    %\else \mapolicebackref{#1 citations pages #2}
%\fi
%}

%R Géneaux - reformattage des backref
\renewcommand*{\backref}[1]{}
\renewcommand*{\backrefalt}[4]{[{\tiny%
    \ifcase #1 Non cité.%
          \or Cité page~#2.%
          \else Cité pages #2.%
    \fi%
    }]}
%redéfinir les séparateurs en français :
\renewcommand*{\backreftwosep}{ et~}
\renewcommand*{\backreflastsep}{, et~}
\renewcommand{\betweenauthors}{et}

% Change URL col
%\def\mybibdoicolor{\color{blue!75!black}} %change color to suit.


%%%%%%%%%%%%%%%%%%%%%%%%%%%%%%%%%%%%%%%%%%%%%%%%
% Define environment for dedication
\newenvironment{narrow}[2]{%
    \begin{list}{}{%
        \setlength{\topsep}{0pt}%
        \setlength{\leftmargin}{#1}%
        \setlength{\rightmargin}{#2}%
        \setlength{\listparindent}{\parindent}%
        \setlength{\itemindent}{\parindent}%
        \setlength{\parsep}{\parskip}}%
        \item[]}{
    \end{list}
}


%%%%%%%%%%%%%%%%%%%%%%%%%%%%%%%%%%%%%%%%%%%%%%%%
% Figures and captions

% Add rules to figures
%\newcommand{\topfigrule}{\vspace*{5pt}{\hrule height0.6pt}\vspace{-5.8pt}}
%\newcommand{\bottomfigrule}{\vspace*{-5.8pt}{\hrule height0.6pt}\vspace{5pt}}

% Select colour or black and white figure, if existing, otherwise the other one.
%\newcommand{\selectgraphics}[1]{
%\ifthenelse{\equal{\colorgraphics}{y}}{\IfFileExists{Figures/#1.pdf}{Figures/#1}{Figures_b_w/#1}}{\IfFileExists{Figures_b_w/#1.pdf}{Figures_b_w/#1}{Figures/#1}}}



% Figure covering the hole page width
\newcommand{\figureinpage}[2]{
%\begin{adjustwidth*}{}{-\marginwidth}
  \begin{figure}[tb]
    %\begin{center}

  %\begin{adjustwidth*}{}{-\marginwidth}
   \begin{adjustwidth*}{}{-4cm}
 \noindent
        \ifthenelse{\equal{\makepreprint}{y}}{
            \includegraphics[width=1\mytextwidth]{Figures/#1}
        }{
            \includegraphics[clip=true,width=1\textwidth+4cm, height=0.99\textheight, keepaspectratio=true]{Figures/#1}
        }
      %\end{center}
    \colfigcaptionpage{#2} \label{fig:#1}
    \end{adjustwidth*}
  \end{figure}

}

\newcommand{\colfigcaptionpage}[1]{
    \changecaptionwidth
    \captionwidth{1\textwidth+4cm}
    \captionnamefont{\small\sffamily\bfseries}
    \captiondelim{. }
    \captionstyle{\linespread{0.8}}
    \captiontitlefont{\small\rmfamily\slshape}
    \setlength{\belowcaptionskip}{0.5\onelineskip}
    \setlength{\abovecaptionskip}{0.5\onelineskip}
    \caption{#1}
}





% Figure in the text
\newcommand{\figureintext}[3]{
  \begin{figure}[#3]
    \begin{center}
        \ifthenelse{\equal{\makepreprint}{y}}{
            \includegraphics[width=1\mytextwidth]{Figures/#1}
        }{
            \includegraphics[width=1\textwidth]{Figures/#1}
          }
      \end{center}
    \colfigcaption{#2} \label{fig:#1}
  \end{figure}
}

\newcommand{\colfigcaption}[1]{
    \changecaptionwidth
    \captionwidth{1\textwidth}
    \captionnamefont{\small\sffamily\bfseries}
    \captiondelim{. }
    \captionstyle{\linespread{0.8}}
    \captiontitlefont{\small\rmfamily\slshape}
    \setlength{\belowcaptionskip}{0.5\onelineskip}
    \setlength{\abovecaptionskip}{0.5\onelineskip}
    \caption{#1}
}

%% Figure in the margin
\newcommand{\figureinmarg}[2]{
    \ifthenelse{\equal{\makepreprint}{y}}{
        \begin{figure}
          \begin{center}
            \includegraphics[width=1\mymarginwidth]{Figures/#1}
          \end{center}
          \colfigcaption{#2}
          \label{fig:#1}
        \end{figure}
    }{
        \sidebar{
          %\parbox{\marginparwidth}{  %%maybe a minipage produecs less artefacts than a parbox...but maybe they're the same 
          \begin{minipage}{\marginparwidth}
            %\begin{center}
              \includegraphics[width=1\marginparwidth]{Figures/#1}
            %\end{center}
            \margfigcaption{#2}
            \label{fig:#1}
          \end{minipage}
          %}
        }
      }
}

\newcommand{\marginalfigcaption}[1]{
    \changecaptionwidth
    \captionwidth{1\marginparwidth}
    \captionnamefont{\footnotesize\sffamily}%\bfseries} %A Camper
    \captiondelim{. }
    \captionstyle{\linespread{0.9}\raggedright}
    \captiontitlefont{\footnotesize\rmfamily\slshape}
    \setlength{\belowcaptionskip}{6pt}
    \setlength{\abovecaptionskip}{3pt}
    \caption{#1}
}

\newfixedcaption[\marginalfigcaption]{\margfigcaption}{figure}

% Sidebar size
\setlength{\sidebarhsep}{\marginparsep}
\setlength{\sidebarvsep}{0pt}
\setlength{\sidebarwidth}{\marginparwidth}
\setsidebarheight{56\onelineskip}
%\setsidebarheight{\textheight}

% Fill out sidebar at new chapters
\newcommand{\fixchaptersidebar}{
\sidebar{\vspace{220pt}}
}

%%%%%%%%%%%%%%%%%%%%%%%%%%%%%%%%%%%%%%%%%%%%%%%%%%
% Prevent overfull lines by inserting extra spaces
\sloppy

% Default mode, no extra spaces and possibly overfull boxes...
%\fussy

%%%%%%%%%%%%%%%%%%%%%%%%%%%%%%%%%%%%%%%%%%%%%%%%%%
% Section numbering and ToC
\maxsecnumdepth{subsection}
\renewcommand{\contentsname}{Sommaire}
\renewcommand{\aftertoctitle}{\thispagestyle{empty}\markboth{\contentsname}{\contentsname}\afterchaptertitle}
\maxtocdepth{subsection}

\setlength{\cftbeforechapterskip}{8pt}


%%%%%%%%%%%%%%%%%%%%%%%%%%%%%%%%%%%%%%%%%%%%%%%%%%
% Select styles
\pagestyle{mypagestyle}
%\chapterstyle{mychapterstyle}
\chapterstyle{thatcher}
%\chapterstyle{veelo}
\strictpagechecktrue


\titleclass{\subsubsubsection}{straight}[\subsection]

\newcounter{subsubsubsection}[subsubsection]
\renewcommand\thesubsubsubsection{\thesubsubsection.\arabic{subsubsubsection}}
\renewcommand\theparagraph{\thesubsubsubsection.\arabic{paragraph}} % optional; useful if paragraphs are to be numbered

\titleformat{\subsubsubsection}
  {\normalfont\normalsize\bfseries}{\thesubsubsubsection}{1em}{}
\titlespacing*{\subsubsubsection}
{6pt}{3.25ex plus 1ex minus .2ex}{1.5ex plus .2ex}

\makeatletter
\renewcommand\paragraph{\@startsection{paragraph}{5}{\z@}%
  {3.25ex \@plus1ex \@minus.2ex}%
  {-1em}%
  {\normalfont\normalsize\bfseries}}
\renewcommand\subparagraph{\@startsection{subparagraph}{6}{\parindent}%
  {3.25ex \@plus1ex \@minus .2ex}%
  {-1em}%
  {\normalfont\normalsize\bfseries}}
\def\toclevel@subsubsubsection{4}
\def\toclevel@paragraph{5}
\def\toclevel@paragraph{6}
\def\l@subsubsubsection{\@dottedtocline{4}{7em}{4em}}
\def\l@paragraph{\@dottedtocline{5}{10em}{5em}}
\def\l@subparagraph{\@dottedtocline{6}{14em}{6em}}
\makeatother

\setcounter{secnumdepth}{5}
\setcounter{tocdepth}{3}

%remove indents at beginning of paragraphs
\setlength{\parindent}{0pt}
\setlength{\parskip}{\baselineskip}