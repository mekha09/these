%%%%%%%%%%%%%%%%%%%%%%%%%%%%%%%%%%%%%%%%%%%%%%%
% Math
\def\rme{\mathrm{e}}
\def\rmc{\mathrm{c}}
\def\rmd{\mathrm{d}}
\def\rmi{\mathrm{i}}
\def\rmx{\mathrm{x}}
\def\rmy{\mathrm{y}}
\def\rmz{\mathrm{z}}
\def\rmg{\mathrm{g}}
\def\rmu{\mathrm{u}}
\def\bfr{\textbf{\em r}}
\def\bfk{\textbf{\em k}}
\def\Ip{I_\mathrm{p}}
\def\cJ{\mathcal{J}}
\def\cL{\mathcal{L}}
\def\cS{\mathcal{S}}
\def\mathbi#1{\textbf{\em #1}}
\def\bfgreek#1{\mbox{\boldmath{$#1$}}}
\def\tensor#1{\overset{\text{\tiny$\bm{\leftrightarrow}$}}{#1}}
\def\sf6{$\text{SF}_\text{6}$}
\def\abs#1{\left|#1\right|^2}

%\def\EXUV{\mbox{${ E}_{\mbox{{\tiny {XUV}}}}$}}
\def\EXUV{\mathbi{E}_{\text{\tiny{XUV}}}}
\def\epsXUV{\bfgreek{\epsilon}_{\text{\tiny{XUV}}}}

%%%%%%%%%%%%%%%%%%%%%%%%%%%%%%%%%%%%%%%%%%%%%%%
% Colors
\definecolor{Gray}{named}{Gray}

%%%%%%%%%%%%%%%%%%%%%%%%%%%%%%%%%%%%%%%%%%%%%%%
% Commenting etc.
\ifthenelse{\equal{\makepreprint}{y}}
{
    \def\comment#1{}
}{
%    \def\comment#1{}
    \def\comment#1{\par\noindent\colorbox{yellow}{\parbox{\textwidth}{\emph{#1}}}}
%  \def\comment#1{\colorbox{yellow}{\emph{#1}}}
}

%%%%%%%%%%%%%%%%%%%%%%%%%%%%%%%%%%%%%%%%%%%%%%%
% Questions to people etc.
\ifthenelse{\equal{\makepreprint}{y}} {
    \def\question#1{}
}{
%    \def\question#1{}
    \def\question#1{\par\noindent\colorbox{Lavender}{\parbox{\textwidth}{\emph{#1}}}}
%    \def\question#1{\par\noindent\colorbox{CornflowerBlue}{\parbox{\textwidth}{\emph{#1}}}}
}

%%%%%%%%%%%%%%%%%%%%%%%%%%%%%%%%%%%%%%%%%%%%%%%
% Cite commands
%\def\mycite#1{(\citet{#1})}  %the way you want to cite your references
%\def\mycite#1{\citet{#1}}  %the way you want to cite your references
\def\mycite#1{\citep{#1}}  %the way you want to cite your references

\def\pre#1{\textbf{\ref{pap:#1}}} %make the paper-numbers bold-face when you cite them

%%%%%%%%%%%%%%%%%%%%%%%%%%%%%%%%%%%%%%%%%%%%%%%
% Roman numerals in text
\newcounter{romancounter}
\newcommand{\rroman}[1]{
  \setcounter{romancounter}{#1}
  \roman{romancounter}
}
\newcommand{\RRoman}[1]{
  \setcounter{romancounter}{#1}
  \Roman{romancounter}
}


%%%%%%%%%%%%%%%%%%%%%%%%%%%%%%%%%%%%%%%%%%%%%%%
% Title
\def\thesistitle{Le moment angulaire de la lumière en génération d'harmoniques d'ordre élevé}

%%%%%%%%%%%%%%%%%%%%%%%%%%%%%%%%%%%%%%%%%%%%%%%
% Set pdf document properties
\hypersetup{
pdfauthor   = {Romain Géneaux <rgeneaux@gmail.com>},%
pdftitle    = {\thesistitle},%
pdfsubject  = {PhD Thesis},%
pdfkeywords = {Keywords},%
}%

%%%%%%%%%%%%%%%%%%%%%%%%%%%%%%%%%%%%%%%%%%%%%%%
% Bibliography name
%\renewcommand{\bibname}{Bibliographie}

%%%%%%%%%%%%%%%%%%%%%%%%%%%%%%%%%%%%%%%%%%%%%%%
% Texts in connection with the List of Papers
\def\loptitle{Liste de Publications}
\def\preloptext{\justify Les articles suivants ont été publiés au cours de cette thèse. Leur version complète est disponible à la fin de ce manuscrit. On y fera référence dans le texte par leur nombre romain. \par\vskip\onelineskip}
\def\postloptext{\par\noindent}

%\def\postloptext{\clearpage \par\noindent Autres publications de l'auteur en lien avec ce sujet :
%\paperlistitem{}
%{Spectrally resolved multi-channel contributions to the harmonic emission in N$_2$}
%{Z. Diveki, A. Camper, S. Haessler, T. Auguste, T. Ruchon, B. Carr\'e, P. Sali\`eres, R. Guichard, J. Caillat, A. Maquet and R. Ta\"ieb }
%{New Journal of Physics}{14}{03062 (2012), DOI:10.1088/1367-2630/14/2/023062}
%\paperlistitem{}
%{Molecular orbital tomography from multi-channel harmonic emission in N$_2$}
%{Z. Diveki, R. Guichard, J. Caillat, A. Camper, S. Haessler, T. Auguste, T. Ruchon, B. Carr\'e, A. Maquet, R. Ta\"ieb and P. Sali\`eres}
%{Chemical Physics}{414}{0121--129 (2013), DOI: 10.1016/j.chemphys.2012.03.021}
%\paperlistitem{}
%{Notes on attosecond pulse profile measurements with the RABBIT technique}
%{T.~Ruchon and A.~Camper}
%{Proceedings of UVX 2012}{\!}{01014 (2013), DOI: 10.1051/uvx/201301014}
%\paperlistitem{}
%{High Harmonic Spectroscopy Using a Binary Diffractive Optical Element}
%{A.~Camper, T.~Ruchon, D.~Gauthier, O.~Gobert, P.~Sali\`eres, B.~Carr\'e and T.~Auguste}
%{Physical Review A}{\!}{submitted (2013)}
%}

%%%%%%%%%%%%%%%%%%%%%%%%%%%%%%%%%%%%%%%%%%%%%%%

%%%%%%%%%%%%%%%%%%%%%%%%%%%%%%%%%%%%%%%%%%%%%%%
% Texts in connection with the Comments on the papers
%\def\loctitle{The Author's Contribution to the Papers}
%\def\loctitle{Contribution de l'auteur aux articles}
%\def\preloctext{}
%\def\postloctext{}
%%%%%%%%%%%%%%%%%%%%%%%%%%%%%%%%%%%%%%%%%%%%%%%

%%%%%%%%%%%%%%%%%%%%%%%%%%%%%%%%%%%%%%%%%%%%%%%
% Own lists with left centered items
\newcommand{\myitem}[4]{
\noindent
\begin{tabular}{p{#3}p{\textwidth-#3-25pt}}
#1 & #2
\end{tabular}
\\[#4]
}

%%%%%%%%%%%%%%%%%%%%%%%%%%%%%%%%%%%%%%%%%%%%%%%
% Synthese-Box for french resumes
\definecolor{gris}{gray}{0.66}
\newcommand{\synthbox}{
	\ifthenelse{\isodd{\thepage}}
	{\sidebar{ \hfill 
	 \colorbox{gris}{\textcolor{white}{ \normalfont\huge \rotatebox{90}{\parbox[b]{3cm}{
                \vspace*{3pt}
                \hspace*{3pt} Synth\`ese \hspace*{3pt}
                \vspace*{6pt}
            }}}}
           %\hspace*{-0.11\foremargin}
  }}
	{\sidebar{ %\hspace*{-0.11\foremargin}
	 \colorbox{gris}{\textcolor{white}{ \normalfont\huge \rotatebox{90}{\parbox[b]{3cm}{
                \vspace*{6pt}
                \hspace*{3pt} Synth\`ese \hspace*{3pt}
                \vspace*{3pt}
            }}}}
   }}
}


%%%%%%%%%%%%%%%%%%%%%%%%%%%%%%%%%%%%%%%%%%%%%%%%%%%%%%%%%%%%%%%%%%%%%%%%%%%%
% Macros for the front page, the second page and the fourth cover (abstract)
% PhD
\newcommand{\PhDTitleFR}{Le moment angulaire de la lumière en génération d'harmoniques d'ordre élevé} %% Titre en Français
\newcommand{\PhDkeywordsFR}{Attoseconde, Interaction laser-molécule, Moment angulaire} %% 3 à 6 mots clefs
\newcommand{\PhDsumFR}{Mon travail de thèse consiste en l'étude de la génération d'harmoniques d'ordre élevé (GHOE), qui permet la génération d'impulsions sub-femtosecondes, dites attosecondes (1as = $10^{-18}$ s). Plus particulièrement, j'étudie l'influence du moment angulaire de la lumière sur ce processus. Cette étude s'articule autour de deux axes, le moment angulaire de spin de la lumière, associé à la polarisation circulaire de la lumière, ainsi que le moment orbital, qui est lui relié à une structure hélicoïdale du front d'onde.} %% Résumé en Français

\newcommand{\PhDTitleEN}{The angular momentum of light in high harmonic generation
} %% Title in English
\newcommand{\PhDkeywordsEN}{Attosecond, Laser-Molecule Interaction, Angular momentum} %% 3-6 Keywords
\newcommand{\PhDsumEN}{My PhD is centered around the study of High Harmonic Generation (HHG), which allows the generation of sub-femtosecond pulses, also called attosecond (1as = $10^{-18}$ s) pulses. More specifically, I study the influence of the angular momentum of light on this process. This study is divided into to parts, first the spin angular momentum of light, associated with the circular polarization of light, and second the orbital angular momentum, which is linked to a helicoidal structure of the light's wavefront.} %% Summary
 
% Personal Information
\newcommand{\PhDname}{M. Romain Géneaux} % Civility, first name and name 
\newcommand{\NNT}{2016SACLS474} %% Your NNT numer (the Library will attribute one...)
\newcommand{\ecodocnum}{572} % Doctoral School number
\newcommand{\ecodoctitle}{Ondes et matière} % Full name of the doctoral school
\newcommand{\PhDspeciality}{Lasers, molécules, rayonnement atmosphérique} % Speciality 
\newcommand{\PhDworkingplace}{l'Université Paris-Sud} %PhD working place (must be one of these : Université Paris-Sud, Université de Versailles Saint Quentin, Université d’Evry, AgroParisTech, CentraleSupelec, Ecole Normale Supérieure de Cachan, Ecole Polytechnique, ENSTA ParisTech,ENSAE ParisTech, HEC,Institut d’Optique Graduate School,Telecom ParisTech or Telecom SudParis)
\newcommand{\laboratory}{Laboratoire Interactions, Dynamiques et Lasers} %Your laboratory
\newcommand{\defenseplace}{Gif-sur-Yvette} %Place of defense
\newcommand{\defensedate}{13 décembre 2016} % Date of defense
\newcommand{\logoED}{\includegraphics[height=2.4cm]{Figures/Logos/EDOM.png}} % Logo of doctoral school. Check the name of the correct file in the /logo folder.
\newcommand{\logoEt}{\includegraphics[height=3cm]{Figures/Logos/UPS.png}} % Must be the logo corresponding to %PhD working place. Check the name of the correct file in the /logo folder.
\newcommand{\vpos}{1} %% You can modify vertical position (leave cm as unit) in order to align horizontally both images

%%%%%%%%%%%%%%%%%%%%%%%%%%%%%%%%%%%%%%%JURY%%%%%%%%%%%%%%%%%%%%%%%%%%%%%%%%%%%%%%%%%%%%%%%%%%%%%%%%%%%%%
% Jury
%%% Jury member n1 (Président) %%%
\newcommand{\jurynameA}{Pierre Agostini}
\newcommand{\jurygenderA}{M. } % M. or Mme. / Mrs.
\newcommand{\juryadressA}{Université d'\'{E}tat de l'Ohio}
\newcommand{\jurygradeA}{Professeur}
\newcommand{\juryroleA}{Rapporteur} % 
%%% Jury member n2 (Rapporteur) %%%
\newcommand{\jurynameB}{Giovanni de Ninno}
\newcommand{\jurygenderB}{M. } % M. or Mme. / Mrs.
\newcommand{\juryadressB}{Université de Nova Gorica}
\newcommand{\jurygradeB}{Professeur}
\newcommand{\juryroleB}{Rapporteur}
%%% Jury member n3 (Rapporteur) %%%
\newcommand{\jurynameC}{Sophie Kazamias}
\newcommand{\jurygenderC}{Mme. } % M. or Mme. / Mrs.
\newcommand{\juryadressC}{Université Paris-Sud}
\newcommand{\jurygradeC}{Professeur}
\newcommand{\juryroleC}{Présidente}
%%% Jury member n4 (Examinateur) %%%
\newcommand{\jurynameD}{Laurent Nahon}
\newcommand{\jurygenderD}{M. } % M. or Mme. / Mrs.
\newcommand{\juryadressD}{Synchrotron Soleil}
\newcommand{\jurygradeD}{Responsable de groupe}
\newcommand{\juryroleD}{Examinateur}
%%% Jury member n5 (Examinateur) %%%
\newcommand{\jurynameE}{Antonio Zelaquett Khoury}
\newcommand{\jurygenderE}{M. } % M. or Mme. / Mrs.
\newcommand{\juryadressE}{Université Fédérale Fluminense}
\newcommand{\jurygradeE}{Professeur}
\newcommand{\juryroleE}{Examinateur}
%%% Jury member n6 (Examinateur) %%%
\newcommand{\jurynameF}{Thierry Ruchon}
\newcommand{\jurygenderF}{M. } % M. or Mme. / Mrs.
\newcommand{\juryadressF}{CEA Saclay}
\newcommand{\jurygradeF}{Chargé de recherche}
\newcommand{\juryroleF}{Encadrant de thèse}
%%%%
\newcommand{\jurynameG}{Bertrand Carré}
\newcommand{\jurygenderG}{M. } % M. or Mme. / Mrs.
\newcommand{\juryadressG}{CEA Saclay}
\newcommand{\jurygradeG}{Responsable de groupe}
\newcommand{\juryroleG}{Directeur de thèse}

% size of CEA logos on second page
\newlength{\logocealength}
\setlength{\logocealength}{3cm}