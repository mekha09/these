%%%%%%%%%%%%%%%%%%%%%%%%%%%%%%%%%%%%%%%%%%%%%%%
% Math
\def\rme{\mathrm{e}}
\def\rmd{\mathrm{d}}
\def\rmi{\mathrm{i}}
\def\rmx{\mathrm{x}}
\def\rmy{\mathrm{y}}
\def\rmz{\mathrm{z}}
\def\rmg{\mathrm{g}}
\def\rmu{\mathrm{u}}
\def\bfr{\textbf{\em r}}
\def\bfk{\textbf{\em k}}
\def\Ip{I_\mathrm{p}}

\def\mathbi#1{\textbf{\em #1}}
\def\bfgreek#1{\mbox{\boldmath{$#1$}}}
 
\def\abs#1{\left|#1\right|^2}

%\def\EXUV{\mbox{${ E}_{\mbox{{\tiny {XUV}}}}$}}
\def\EXUV{\mathbi{E}_{\text{\tiny{XUV}}}}
\def\epsXUV{\bfgreek{\epsilon}_{\text{\tiny{XUV}}}}

%%%%%%%%%%%%%%%%%%%%%%%%%%%%%%%%%%%%%%%%%%%%%%%
% Colors
\definecolor{Gray}{named}{Gray}

%%%%%%%%%%%%%%%%%%%%%%%%%%%%%%%%%%%%%%%%%%%%%%%
% Commenting etc.
\ifthenelse{\equal{\makepreprint}{y}}
{
    \def\comment#1{}
}{
%    \def\comment#1{}
    \def\comment#1{\par\noindent\colorbox{yellow}{\parbox{\textwidth}{\emph{#1}}}}
%  \def\comment#1{\colorbox{yellow}{\emph{#1}}}
}

%%%%%%%%%%%%%%%%%%%%%%%%%%%%%%%%%%%%%%%%%%%%%%%
% Questions to people etc.
\ifthenelse{\equal{\makepreprint}{y}} {
    \def\question#1{}
}{
%    \def\question#1{}
    \def\question#1{\par\noindent\colorbox{Lavender}{\parbox{\textwidth}{\emph{#1}}}}
%    \def\question#1{\par\noindent\colorbox{CornflowerBlue}{\parbox{\textwidth}{\emph{#1}}}}
}

%%%%%%%%%%%%%%%%%%%%%%%%%%%%%%%%%%%%%%%%%%%%%%%
% Cite commands
%\def\mycite#1{(\citet{#1})}  %the way you want to cite your references
%\def\mycite#1{\citet{#1}}  %the way you want to cite your references
\def\mycite#1{\citep{#1}}  %the way you want to cite your references

\def\pre#1{\textbf{\ref{pap:#1}}} %make the paper-numbers bold-face when you cite them

%%%%%%%%%%%%%%%%%%%%%%%%%%%%%%%%%%%%%%%%%%%%%%%
% Roman numerals in text
\newcounter{romancounter}
\newcommand{\rroman}[1]{
  \setcounter{romancounter}{#1}
  \roman{romancounter}
}
\newcommand{\RRoman}[1]{
  \setcounter{romancounter}{#1}
  \Roman{romancounter}
}


%%%%%%%%%%%%%%%%%%%%%%%%%%%%%%%%%%%%%%%%%%%%%%%
% Title
\def\thesistitle{Le Titre}
%\def\thesistitle{Tomographie d'orbitales moléculaires résolue en temps à l'échelle attoseconde}

%%%%%%%%%%%%%%%%%%%%%%%%%%%%%%%%%%%%%%%%%%%%%%%
% Set pdf document properties
\hypersetup{
pdfauthor   = {Romain Géneaux <romain.geneaux@cea.fr>},%
pdftitle    = {\thesistitle},%
pdfsubject  = {PhD Thesis},%
pdfkeywords = {Keywords},%
}%

%%%%%%%%%%%%%%%%%%%%%%%%%%%%%%%%%%%%%%%%%%%%%%%
% Bibliography name
%\renewcommand{\bibname}{Bibliographie}

%%%%%%%%%%%%%%%%%%%%%%%%%%%%%%%%%%%%%%%%%%%%%%%
% Texts in connection with the List of Papers
\def\loptitle{Liste de Publications}
%\def\preloptext{Cette thèse est construite à partir des articles suivants, auxquels il sera fait référence dans le texte par leur nombre romain.\\}
\def\postloptext{\par\noindent}

%\def\postloptext{\clearpage \par\noindent Autres publications de l'auteur en lien avec ce sujet :
%\paperlistitem{}
%{Spectrally resolved multi-channel contributions to the harmonic emission in N$_2$}
%{Z. Diveki, A. Camper, S. Haessler, T. Auguste, T. Ruchon, B. Carr\'e, P. Sali\`eres, R. Guichard, J. Caillat, A. Maquet and R. Ta\"ieb }
%{New Journal of Physics}{14}{03062 (2012), DOI:10.1088/1367-2630/14/2/023062}
%\paperlistitem{}
%{Molecular orbital tomography from multi-channel harmonic emission in N$_2$}
%{Z. Diveki, R. Guichard, J. Caillat, A. Camper, S. Haessler, T. Auguste, T. Ruchon, B. Carr\'e, A. Maquet, R. Ta\"ieb and P. Sali\`eres}
%{Chemical Physics}{414}{0121--129 (2013), DOI: 10.1016/j.chemphys.2012.03.021}
%\paperlistitem{}
%{Notes on attosecond pulse profile measurements with the RABBIT technique}
%{T.~Ruchon and A.~Camper}
%{Proceedings of UVX 2012}{\!}{01014 (2013), DOI: 10.1051/uvx/201301014}
%\paperlistitem{}
%{High Harmonic Spectroscopy Using a Binary Diffractive Optical Element}
%{A.~Camper, T.~Ruchon, D.~Gauthier, O.~Gobert, P.~Sali\`eres, B.~Carr\'e and T.~Auguste}
%{Physical Review A}{\!}{submitted (2013)}
%}

%%%%%%%%%%%%%%%%%%%%%%%%%%%%%%%%%%%%%%%%%%%%%%%

%%%%%%%%%%%%%%%%%%%%%%%%%%%%%%%%%%%%%%%%%%%%%%%
% Texts in connection with the Comments on the papers
%\def\loctitle{The Author's Contribution to the Papers}
%\def\loctitle{Contribution de l'auteur aux articles}
%\def\preloctext{}
%\def\postloctext{}
%%%%%%%%%%%%%%%%%%%%%%%%%%%%%%%%%%%%%%%%%%%%%%%

%%%%%%%%%%%%%%%%%%%%%%%%%%%%%%%%%%%%%%%%%%%%%%%
% Own lists with left centered items
\newcommand{\myitem}[4]{
\noindent
\begin{tabular}{p{#3}p{\textwidth-#3-25pt}}
#1 & #2
\end{tabular}
\\[#4]
}

%%%%%%%%%%%%%%%%%%%%%%%%%%%%%%%%%%%%%%%%%%%%%%%
% Synthese-Box for french resumes
\definecolor{gris}{gray}{0.66}
\newcommand{\synthbox}{
	\ifthenelse{\isodd{\thepage}}
	{\sidebar{ \hfill 
	 \colorbox{gris}{\textcolor{white}{ \normalfont\huge \rotatebox{90}{\parbox[b]{3cm}{
                \vspace*{3pt}
                \hspace*{3pt} Synth\`ese \hspace*{3pt}
                \vspace*{6pt}
            }}}}
           %\hspace*{-0.11\foremargin}
  }}
	{\sidebar{ %\hspace*{-0.11\foremargin}
	 \colorbox{gris}{\textcolor{white}{ \normalfont\huge \rotatebox{90}{\parbox[b]{3cm}{
                \vspace*{6pt}
                \hspace*{3pt} Synth\`ese \hspace*{3pt}
                \vspace*{3pt}
            }}}}
   }}
}